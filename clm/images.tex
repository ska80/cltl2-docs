\batchmode
\documentstyle[twoside,permutemonospace]{CLTL}
\makeatletter
%

\sloppy


\ifchiron
  \let\relax
\else
  
\fi



\let\newer=\new

\pagestyle{headings}

\makeindex

%























\def\pagestatus{\error}

\def\tagline{\ifcase\month\or
    January\or February\or March\or April\or May\or June\or
    July\or August\or September\or October\or November\or December\fi~\number\day,~\number\year
    ---\pagestatus---Common Lisp: The Language 2/E---Digital Press---\copyright~\number\year
    ~Guy L. Steele Jr. All rights reserved.}

\def\tagline{}

\def\pagestatus{ROUGH PAGES}

\def\numberline#1{\setbox0=\hbox{#1 }\ifdim \wd0 < \@tempdima
    \hbox to \@tempdima{\box0\hfil}\else \box0 \fi}

\def\pagestatus{FINAL PROOF}

\def\GrossOptVars{\Mstar{{\it var} {\Mor} \cd{(}{\it var} \Mopt{{\it initform} \Mopt{{\it svar}}}\cd{)}}}

\def\GrossKeyOptVars{\Mstar{{\it var} {\Mor} \cd{(}\Mgroup{{\it var} {\Mor} \cd{(}{\it keyword} {\it var}\cd{)}} \Mopt{{\it initform} \Mopt{{\it svar}}}\cd{)}}}

\def\GrossAuxVars{\Mstar{{\it var} {\Mor} \cd{(}{\it var} \Mopt{{\it initform}}\cd{)}}}

\def\pagestatus{ULTIMATE}

\def\arcsinh{\mathop{\rm arcsinh}\nolimits}

\def\arccosh{\mathop{\rm arccosh}\nolimits}

\def\arctanh{\mathop{\rm arctanh}\nolimits}

\def\cis{\mathop{\rm cis}\nolimits}

\def\phase{\mathop{\rm phase}\nolimits}

\def\numplot#1{\vskip -2pt \vskip 28pc \PostScriptFile{#1-plot.ps}\vskip 3pt}

\def\arraystretch{1.1}

\def\foo{-6pt}

\def\@listi{\leftmargin\leftmargini \labelsep\leftmargin
   \parsep 3pt\relax
   \topsep 2pt plus 5pt\relax
   \itemsep\topsep}

\def\CLOS{Common Lisp Object System}

\def\OS{Object System}

\def\bit{\it}

\def\sub{_}

\def\Qhyphen{\copy\Qhyphbox}

\def\foohyphen{\copy\hyphbox}

\def\SU#1{\({}_{#1}\)}

\def\@listi{\leftmargin\leftmargini \labelsep\leftmargin
   \parsep 3pt\relax
   \topsep 4pt plus 9pt\relax
   \itemsep\topsep}

\def\fooprime#1{#1'}

\def\foo#1#2#3{\vbox to 0pt{\vskip #2\vskip 3pt\hbox to 0pt{\hskip #1\hskip -3pt\vbox to 0pt{\vss
   \hbox to 0pt{\hss \tt #3\hss}\vss}\hss}\vss}}

\def\@listi{\leftmargin\leftmargini \labelsep\leftmargin
   \parsep 3pt\relax
   \topsep 4pt plus 10pt\relax
   \itemsep\topsep}
\newdimen\foodimen
\foodimen=\textheight
\newbox\Qhyphbox
\newbox\hyphbox

\makeatother
\newenvironment{tex2html_wrap}{}{}
\begin{document}
\pagestyle{empty}
\stepcounter{chapter}
\stepcounter{section}
\stepcounter{section}
\stepcounter{subsection}
\stepcounter{subsection}
{\newpage
\clearpage
\samepage \EV
}

\stepcounter{subsection}
{\newpage
\clearpage
\samepage \EX
}

{\newpage
\clearpage
\samepage \EQ
}

\stepcounter{subsection}
{\newpage
\clearpage
\samepage \begin{table}% latex2html id marker 164
[t]
\caption{Sample Function Description}
\label{Sample-Function-Description}
\begingroup\normalsize
\begin{defun}*[Function]
sample-function arg1 arg2 &optional arg3 arg4

The function \cd{sample-function} adds together {\it arg1} and {\it arg2}, and
then multiplies the result by {\it arg3}.  If {\it arg3} is not provided or
is {\nil}, the multiplication isn't done.  \cd{sample-function} then returns
a list whose first element is this result and whose second element is
{\it arg4} (which defaults to the symbol \cd{foo}).
For example:
\begin{lisp}
(sample-function 3 4) \EV\ (7 foo) \\ 
(sample-function 1 2 2 'bar) \EV\ (6 bar)
\end{lisp}
In general,
\cd{(sample-function {\it x} {\it y})} \EQ\ \cd{(list (+ {\it x} {\it y}) 'foo)}.
\end{defun}
\endgroup
\vskip\ruletonoteskip
\hrule
\vskip\ruletonoteskip\null
\caption{Sample Variable Description}
\label{Sample-Variable-Description}
\begingroup\normalsize
\begin{defun}*[Variable]
*sample-variable*

The variable \cd{*sample-variable*} specifies how many times
the special form \cd{sample-special-form} should iterate.
The value should always be a non-negative integer or {\nil}
(which means iterate indefinitely many times).  The initial value is \cd{0}
(meaning no iterations).
\end{defun}
\endgroup
\vskip\ruletonoteskip
\hrule
\vskip\ruletonoteskip\null
\caption{Sample Constant Description}
\label{Sample-Constant-Description}
\begingroup\normalsize
\begin{defun}*[Constant]
sample-constant

The named constant \cd{sample-constant} has as its value
the height of the terminal screen in furlongs times
the base-2 logarithm of the implementation's total disk capacity in bytes,
as a floating-point number.
\end{defun}
\endgroup
\end{table}
}

{\newpage
\clearpage
\samepage \begin{table}% latex2html id marker 198
[t]
\caption{Sample Special Form Description}
\label{Sample-Special-Form-Description}
\begingroup\normalsize
\begin{defspec}*
sample-special-form [name] ({var}*) {\,form}+

This evaluates each form in sequence as an implicit \cd{progn}, and does this
as many times as specified by
the global variable \cd{*sample-variable*}.  Each variable {\it var} is bound
and initialized to \cd{43} before the first iteration, and unbound after
the last iteration.
The name {\it name}, if supplied, may be used in a \cd{return-from} form
to exit from the loop prematurely.  If the loop ends normally,
\cd{sample-special-form} returns {\nil}.
For example:
\begin{lisp}
(setq *sample-variable* 3) \\ 
(sample-special-form {\empty} {\it form1} {\it form2})
\end{lisp}
This evaluates {\it form1}, {\it form2}, {\it form1}, {\it form2}, {\it form1}, {\it form2}
in that order.
\end{defspec}
\endgroup
\vskip\ruletonoteskip
\hrule
\vskip\ruletonoteskip\null
\caption{Sample Macro Description}
\label{Sample-Macro-Description}
\begingroup\normalsize
\begin{defmac}*
sample-macro var <declaration* | doc-string> {tag | statement}*

This evaluates the statements as a \cd{prog} body,
with the variable {\it var} bound to \cd{43}.
\begin{lisp}
(sample-macro x (return (+ x x))) \EV\ 86 \\ 
(sample-macro {\it var} . {\it body}) \EX\ (prog (({\it var} 43)) . {\it body})
\end{lisp}
\end{defmac}
\endgroup
\end{table}
}

\stepcounter{subsection}
{\newpage
\clearpage
\samepage ${}^*$
}

{\newpage
\clearpage
\samepage ${}^+$
}

\stepcounter{subsection}
\stepcounter{subsection}
{\newpage
\clearpage
\samepage \Xquote
}

{\newpage
\clearpage
\samepage $105_{8}$
}

{\newpage
\clearpage
\samepage $105_{16}$
}

{\newpage
\clearpage
\samepage $1011_{2}$
}

\stepcounter{chapter}
\stepcounter{section}
\stepcounter{subsection}
{\newpage
\clearpage
\samepage $-2^{\hbox{\scriptsize\it n}}$
}

{\newpage
\clearpage
\samepage $2^{\hbox{\scriptsize\it n}}-1$
}

{\newpage
\clearpage
\samepage \(e^{\pi i}\)
}

{\newpage
\clearpage
\samepage \(-192\)
}

\stepcounter{subsection}
{\newpage
\clearpage
\samepage \((-5/2)^{15}\)
}

\stepcounter{subsection}
{\newpage
\clearpage
\samepage ${\it s} \cdot {\it f} \cdot {\it b}^{\hbox{\scriptsize\it e}-\hbox{\scriptsize\it p}}$
}

{\newpage
\clearpage
\samepage ${\it b}^{\,\hbox{\scriptsize\it p}-1}$
}

{\newpage
\clearpage
\samepage ${\it p} \log_2 {\it b}$
}

{\newpage
\clearpage
\samepage \begin{table}% latex2html id marker 670
[t]
\caption{Recommended Minimum Floating-Point Precision and Exponent Size}
\label{Floating-Format-Requirements-Table}
\begin{tabular}{@{}lll@{}}
{Format\quad\quad}&{Minimum Precision\quad\quad}&{Minimum Exponent Size} \\  \hlinesp
Short&13 bits&5 bits \\ 
Single&24 bits&8 bits \\ 
Double&50 bits&8 bits \\ 
Long&50 bits&8 bits
\end{tabular}
\end{table}
}

{\newpage
\clearpage
\samepage \(\pi\)
}

{\newpage
\clearpage
\samepage \(\log_{10} 2\)
}

\stepcounter{subsection}
\stepcounter{section}
\stepcounter{subsection}
{\newpage
\clearpage
\samepage \Xcircumflex
}

\stepcounter{subsection}
\stepcounter{subsection}
{\newpage
\clearpage
\samepage % latex2html id marker 40236
\(\alpha\)
}

\stepcounter{subsection}
\stepcounter{subsection}
\stepcounter{section}
\stepcounter{section}
\stepcounter{section}
\stepcounter{subsection}
\stepcounter{subsection}
\stepcounter{subsection}
\stepcounter{section}
\stepcounter{section}
\stepcounter{section}
\stepcounter{section}
\stepcounter{section}
\stepcounter{section}
\stepcounter{section}
\stepcounter{section}
\stepcounter{section}
\stepcounter{section}
{\newpage
\clearpage
\samepage ${\it a}_1$
}

{\newpage
\clearpage
\samepage ${\it a}_{\hbox{\scriptsize\it n}}$
}

{\newpage
\clearpage
\samepage ${\it a}_j$
}

{\newpage
\clearpage
\samepage ${\it a}_{\hbox{\scriptsize\it j}}$
}

\stepcounter{chapter}
{\newpage
\clearpage
\samepage \({}_1\)
}

{\newpage
\clearpage
\samepage \({}_2\)
}

\stepcounter{chapter}
\stepcounter{section}
\stepcounter{section}
{\newpage
\clearpage
\samepage \begin{table}% latex2html id marker 2096
[t]
\caption{Standard Type Specifier Symbols}
\label{TYPE-SYMBOLS-TABLE}
\divide\tabcolsep by 2\relax
\begin{flushleft}
\cf
\begin{tabular*}{\textwidth}{@{}l@{\extracolsep{\fill}}l@{\extracolsep{\fill}}l@{\extracolsep{\fill}}l@{}}
array&fixnum&package&simple-string \\ 
atom&float&pathname&simple-vector \\ 
bignum&function&random-state&single-float \\ 
bit&hash-table&ratio&standard-char \\ 
bit-vector&integer&rational&stream \\ 
character&keyword&readtable&string \\ 
{\rm [}common{\rm ]}&list&sequence&{\rm [}string-char{\rm ]} \\ 
compiled-function&long-float&short-float&symbol \\ 
complex&nil&signed-byte&t \\ 
cons&null&simple-array&unsigned-byte \\ 
double-float&number&simple-bit-vector&vector
\end{tabular*}
\end{flushleft}

\begin{newer}
X3J13 voted in March 1989 \issue{COMMON-TYPE} to remove the type \cd{common}.

X3J13 voted in March 1989 \issue{CHARACTER-PROPOSAL} to remove the type \cd{string-char}.

X3J13 voted in March 1989 \issue{CHARACTER-PROPOSAL}
to add \cd{base-character} and \cd{extended-character}.

X3J13 voted in March 1989 \issue{REAL-NUMBER-TYPE} to add the type \cd{real}.
\end{newer}
\end{table}
}

\stepcounter{section}
\stepcounter{section}
\stepcounter{section}
\stepcounter{section}
{\newpage
\clearpage
\samepage $-2^{\hbox{\scriptsize\it s}-1}$
}

{\newpage
\clearpage
\samepage $2^{\hbox{\scriptsize\it s}-1}-1$
}

{\newpage
\clearpage
\samepage $2^{\hbox{\scriptsize\it s}}$
}

{\newpage
\clearpage
\samepage $2^{\hbox{\scriptsize\it s}}-1$
}

\stepcounter{section}
\stepcounter{section}
\stepcounter{section}
\stepcounter{section}
\stepcounter{chapter}
\stepcounter{section}
\stepcounter{subsection}
\stepcounter{subsection}
\stepcounter{subsection}
{\newpage
\clearpage
\samepage \begin{table}% latex2html id marker 3064
[t]
\caption{Names of All Common Lisp Special Forms}
\label{SPECIAL-FORM-TABLE}
\begin{tabular*}{\textwidth}{@{\extracolsep{\fill}}lll@{}}
\cd{block}&\cd{if}&\cd{progv} \\ 
\cd{catch}&\cd{labels}&\cd{quote} \\ 
{\rm \lbrack}\cd{compiler-let}{\rm \rbrack}&\cd{let}&\cd{return-from} \\ 
\cd{declare}&\cd{let*}&\cd{setq} \\ 
\cd{eval-when}&\cd{macrolet}&\cd{tagbody} \\ 
\cd{flet}&\cd{multiple-value-call}&\cd{the} \\ 
\cd{function}&\cd{multiple-value-prog1}&\cd{throw} \\ 
\cd{go}&\cd{progn}&\cd{unwind-protect}
\end{tabular*}
\vskip 4pt
\begin{newer}
X3J13 voted in June 1989 \issue{COMPILER-LET-CONFUSION} to remove
\cd{compiler-let} from the language.
\end{newer}

\begin{newer}
X3J13 voted in June 1988 \issue{CLOS} to add the special forms \cd{generic-flet},
\cd{generic-labels}, \cd{symbol-macrolet}, and \cd{with-added-methods}.
\end{newer}

\begin{newer}
X3J13 voted in March 1989 \issue{LOCALLY-TOP-LEVEL} to make
\cd{locally} a special form rather than a macro.
\end{newer}

\begin{newer}
X3J13 voted in March 1989 \issue{LOAD-TIME-EVAL} to add the special form
\cd{load-time-eval}.
\end{newer}
\end{table}
}

\stepcounter{subsection}
\stepcounter{subsection}
\stepcounter{section}
\stepcounter{subsection}
\stepcounter{subsection}
\stepcounter{section}
\stepcounter{subsection}
\stepcounter{subsection}
\stepcounter{subsection}
{\newpage
\clearpage
\samepage \begin{tabular*}{\linewidth}{@{\extracolsep{\fill}}c@{}cccl@{}}
     LT&CT&EX&CTTM&Action \\  \hlinesp
       yes & yes &--   & --  &    process body in compile-time-too mode \\ 
       yes & no  &yes  & yes &    process body in compile-time-too mode \\ 
       yes & no  &--   & no  &    process body in not-compile-time mode \\ 
       yes & no  &no   & --  &    process body in not-compile-time mode \\ 
       no  & yes &--   & --  &    evaluate body \\ 
       no  & no  &yes  & yes &    evaluate body \\ 
       no  & no  &--   & no  &    do nothing \\ 
       no  & no  &no   & --  &    do nothing \\ 
       \hline
     \end{tabular*}
}

\stepcounter{chapter}
\stepcounter{section}
\stepcounter{section}
\stepcounter{subsection}
\stepcounter{subsection}
{\newpage
\clearpage
\samepage $\;\equiv\;$
}

\stepcounter{section}
\stepcounter{section}
{\newpage
\clearpage
\samepage \(\ldots\)
}

\stepcounter{chapter}
\stepcounter{section}
\stepcounter{subsection}
\stepcounter{subsection}
\stepcounter{section}
{\newpage
\clearpage
\samepage \begin{tabular*}{\textwidth}{@{}l@{\extracolsep{\fill}}ll@{}}
{\rm Access Function}&{\rm Update Function}&{\rm Update Using \cd{setf}} \\ 
\hlinesp
\cd{x}&\cd{(setq x datum)}&\cd{(setf x datum)} \\ 
\cd{(car x)}&\cd{(rplaca x datum)}&\cd{(setf (car x) datum)} \\ 
\cd{(symbol-value x)}&\cd{(set x datum)}&\cd{(setf (symbol-value x) datum)} \\ 
\hline
\end{tabular*}
}

{\newpage
\clearpage
\samepage ${}_1$
}

{\newpage
\clearpage
\samepage ${}_2$
}

{\newpage
\clearpage
\samepage ${}_k$
}

\stepcounter{section}
\stepcounter{section}
\stepcounter{section}
\stepcounter{section}
\stepcounter{section}
\stepcounter{section}
\stepcounter{subsection}
\stepcounter{subsection}
{\newpage
\clearpage
\samepage ${\it k\/}<{\it j}$
}

\stepcounter{subsection}
\stepcounter{subsection}
\stepcounter{subsection}
\stepcounter{section}
{\newpage
\clearpage
\samepage \begin{table}% latex2html id marker 7685
[t]
\begin{new}
\leavevmode
\vtop{
\caption{Structure Traversal Operations Subject to Side Effect Restrictions}
\label{TRAVERSAL-OPERATIONS-TABLE}
}

\begingroup\cf \tabcolsep0pt\relax
\begin{tabular*}{\textwidth}{@{}l@{\extracolsep{\fill}}ll@{}}
adjoin & maphash & reduce \\ 
assoc & mapl & remove \\ 
assoc-if & maplist & remove-duplicates \\ 
assoc-if-not & member & remove-if \\ 
count & member-if & remove-if-not \\ 
count-if & member-if-not & search \\ 
count-if-not & merge & set-difference \\ 
delete & mismatch & set-exclusive-or \\ 
delete-duplicates & nintersection & some \\ 
delete-if & notany & sort \\ 
delete-if-not & notevery & stable-sort \\ 
do-all-symbols & nset-difference & sublis \\ 
do-external-symbols & nset-exclusive-or & subsetp \\ 
do-symbols & nsublis & subst \\ 
dolist & nsubst & subst-if \\ 
eval & nsubst-if & subst-if-not \\ 
every & nsubst-if-not & substitute \\ 
find & nsubstitute & substitute-if \\ 
find-if & nsubstitute-if & substitute-if-not \\ 
find-if-not & nsubstitute-if-not & tree-equal \\ 
intersection & nunion & union \\ 
{\rm certain} loop {\rm clauses} & position & with-hash-table-iterator \\ 
map & position-if & with-input-from-string \\ 
mapc & position-if-not & with-output-to-string \\ 
mapcan & rassoc & with-package-iterator \\ 
mapcar & rassoc-if \\ 
mapcon & rassoc-if-not
\end{tabular*}
\endgroup
\end{new}
\end{table}
}

\stepcounter{section}
\stepcounter{subsection}
\stepcounter{subsection}
\stepcounter{section}
\stepcounter{chapter}
\stepcounter{section}
\stepcounter{section}
\stepcounter{section}
\stepcounter{section}
\stepcounter{section}
\stepcounter{chapter}
\stepcounter{section}
\stepcounter{section}
{\newpage
\clearpage
\samepage ${\it w}_1, \ldots, {\it w}_{\hbox{\scriptsize\it n}}$
}

{\newpage
\clearpage
\samepage ${\it w}_{\hbox{\scriptsize\it j}}$
}

\stepcounter{section}
\stepcounter{chapter}
\stepcounter{section}
\stepcounter{section}
\stepcounter{section}
\stepcounter{chapter}
\stepcounter{section}
\stepcounter{section}
\stepcounter{section}
\stepcounter{section}
\stepcounter{section}
\stepcounter{section}
\stepcounter{section}
\stepcounter{section}
\stepcounter{section}
{\newpage
\clearpage
\samepage \begin{table}% latex2html id marker 11146
[t]
\caption{An Initialization File}
\label{INIT-FILE-TABLE}
\begin{lisp}
;;;; Lisp init file for I. Newton. \\ 
 \\ 
;;; Set up the USER package the way I like it. \\ 
 \\ 
(require 'calculus)~~~~~~~~~~~~~;I use CALCULUS a lot; load it. \\ 
(use-package 'calculus)~~~~~~~~~;Get easy access to its \\ 
~~~~~~~~~~~~~~~~~~~~~~~~~~~~~~~~; exported symbols. \\ 
 \\ 
(require 'newtonian-mechanics)~~;Same thing for NEWTONIAN-MECHANICS \\ 
(use-package 'newtonian-mechanics) \\ 
 \\ 
;;; I just want a few things from RELATIVITY, \\ 
;;; and other things conflict. \\ 
;;; Import only what I need into the USER package. \\ 
 \\ 
(require 'relativity) \\ 
(import '(relativity:speed-of-light \\ 
~~~~~~~~~~relativity:ignore-small-errors)) \\ 
 \\ 
;;; These are worth loading, but I will use qualified names, \\ 
;;; such as PHLOGISTON:MAKE-FIRE-BOTTLE, to get at any symbols \\ 
;;; I might need from these packages. \\ 
 \\ 
(require 'phlogiston) \\ 
(require 'alchemy) \\ 
 \\ 
;;; End of Lisp init file for I. Newton.
\end{lisp}
\end{table}
}

{\newpage
\clearpage
\samepage \begin{table}% latex2html id marker 11177
\caption{File \protect\cd{alchemy}}
\label{ALCHEMY-FILE-TABLE}
\begin{lisp}
;;;; Alchemy functions, written and maintained by Merlin, Inc. \\ 
 \\ [4pt]
(provide 'alchemy)~~~~~~~~~~~~~~~~~~;The module is named ALCHEMY. \\ 
(in-package 'alchemy)~~~~~~~~~~~~~~~;So is the package. \\ 
 \\ [4pt]
;;; There is nothing to shadow. \\ 
 \\ [4pt]
;;; Here is the external interface. \\ 
 \\ [4pt]
(export '(lead-to-gold gold-to-lead  \\ 
~~~~~~~~~~antimony-to-zinc elixir-of-life)) \\ 
 \\ [4pt]
;;; This package/module needs a function from \\ 
;;; the PHLOGISTON package/module. \\ 
 \\ [4pt]
(require 'phlogiston) \\ 
 \\ [4pt]
;;; We don't frequently need most of the external symbols from \\ 
;;; PHLOGISTON, so it's not worth doing a USE-PACKAGE on it. \\ 
;;; We'll just use qualified names as needed.~~But we use \\ 
;;; one function, MAKE-FIRE-BOTTLE, a lot, so import it. \\ 
;;; It's external in PHLOGISTON and so can be referred to \\ 
;;; here using ":" qualified-name syntax. \\ 
 \\ [4pt]
(import '(phlogiston:make-fire-bottle)) \\ 
 \\ [4pt]
;;; Now for the real contents of this file. \\ 
 \\ [4pt]
(defun lead-to-gold (x) \\ 
~~"Takes a quantity of lead and returns gold." \\ 
~~(when (> (phlogiston:heat-flow 5 x x)~~;Using a qualified symbol \\ 
~~~~~~~~~~~3) \\ 
~~~~(make-fire-bottle x))~~~~~~~~~~~~~~~~;Using an imported symbol \\ 
~~(gild x)) \\ 
 \\ [4pt]
;;; And so on ...
\end{lisp}
\vfill
\end{table}
}

{\newpage
\clearpage
\samepage \begin{table}% latex2html id marker 11183
\caption{File \protect\cd{phlogiston}}
\label{PHLOGISTON-FILE-TABLE}
\begin{lisp}
;;;; Phlogiston functions, by Thermofluidics, Ltd. \\ 
 \\ [4pt]
(provide 'phlogiston)~~~~~~~~~~~~~;The module is named PHLOGISTON. \\ 
(in-package 'phlogiston)~~~~~~~~~~;So is the package. \\ 
 \\ [4pt]
;;; There is nothing to shadow. \\ 
 \\ [4pt]
;;; Here is the external interface. \\ 
 \\ [4pt]
(export '(heat-flow cold-flow mix-fluids separate-fluids \\ 
~~~~~~~~~~burn make-fire-bottle)) \\ 
 \\ [4pt]
;;; This file uses functions from the ALCHEMY package/module. \\ 
 \\ [4pt]
(require 'alchemy) \\ 
 \\ [4pt]
;;; We use alchemy functions a lot, so use the package. \\ 
;;; This will allow symbols exported from the ALCHEMY package \\ 
;;; to be referred to here without the need for qualified names. \\ 
 \\ [4pt]
(use-package 'alchemy) \\ 
 \\ [4pt]
;;; No calls to IMPORT are needed here. \\ 
 \\ [4pt]
;;; The real contents of this package/module. \\ 
\\ [4pt]
(defvar *feeling-weak* nil) \\ 
 \\ [4pt]
(defun heat-flow (amount x y) \\ 
~~"Make some amount of heat flow from x to y." \\ 
~~(when *feeling-weak* \\ 
~~~~(quaff (elixir-of-life)))~~~~~;No qualifier is needed. \\ 
~~(push-heat amount x y)) \\ 
 \\ [4pt]
;;; And so on ...
\end{lisp}
\vfill
\end{table}
}

{\newpage
\clearpage
\samepage \begin{table}% latex2html id marker 11215
[t]
\caption{An Initialization File When \protect\cd{defpackage} Is Used}
\label{DEFPACKAGE-INIT-FILE-TABLE}
\begin{lisp}
;;;; Lisp init file for I. Newton. \\ 
 \\ 
;;; Set up the USER package the way I like it. \\ 
 \\ 
(load "calculus")~~~~~~~~~~~~~~~;I use CALCULUS a lot; load it. \\ 
(use-package 'calculus)~~~~~~~~~;Get easy access to its \\ 
~~~~~~~~~~~~~~~~~~~~~~~~~~~~~~~~; exported symbols. \\ 
 \\ 
(load "newtonian-mechanics")~~~~;Ditto for NEWTONIAN-MECHANICS \\ 
(use-package 'newtonian-mechanics) \\ 
 \\ 
;;; I just want a few things from RELATIVITY, \\ 
;;; and other things conflict. \\ 
;;; Import only what I need into the USER package. \\ 
 \\ 
(load "relativity") \\ 
(import '(relativity:speed-of-light \\ 
~~~~~~~~~~relativity:ignore-small-errors)) \\ 
 \\ 
;;; These are worth loading, but I will use qualified names, \\ 
;;; such as PHLOGISTON:MAKE-FIRE-BOTTLE, to get at any symbols \\ 
;;; I might need from these packages. \\ 
 \\ 
(load "phlogiston") \\ 
(load "alchemy") \\ 
 \\ 
;;; End of Lisp init file for I. Newton.
\end{lisp}
\end{table}
}

{\newpage
\clearpage
\samepage \begin{table}% latex2html id marker 11248
[t]
\caption{File \protect\cd{alchemy-package} Using \protect\cd{defpackage}}
\label{DEFPACKAGE-ALCHEMY-PACKAGE-TABLE}
\begin{lisp}
;;;; Alchemy package, written and maintained by Merlin, Inc. \\ 
 \\ 
(cl:defpackage "ALCHEMY" \\ 
~~(:export "LEAD-TO-GOLD" "GOLD-TO-LEAD" \\ 
~~~~~~~~~~~"ANTIMONY-TO-ZINC" "ELIXIR-OF-LIFE") \\ 
~~) \\ 
 \\ 
;;; This package needs a function from the PHLOGISTON package. \\ 
;;; Load the definition of the PHLOGISTON package if necessary. \\ 
 \\ 
(cl:unless (cl:find-package "PHLOGISTON") \\ 
~~(cl:load "phlogiston-package")) \\ 
 \\ 
;;; We don't frequently need most of the external symbols from \\ 
;;; PHLOGISTON, so it's not worth doing a USE-PACKAGE on it. \\ 
;;; We'll just use qualified names as needed.  But we use \\ 
;;; one function, MAKE-FIRE-BOTTLE, a lot, so import it. \\ 
;;; It's external in PHLOGISTON and so can be referred to \\ 
;;; here using ":" qualified-name syntax. \\ 
 \\ 
(cl:import '(phlogiston:make-fire-bottle))
\end{lisp}
\vskip\ruletonoteskip
\hrule
\vskip\ruletonoteskip\null
\caption{File \protect\cd{alchemy} Using \protect\cd{defpackage}}
\label{DEFPACKAGE-ALCHEMY-FILE-TABLE}
\begin{lisp}
;;;; Alchemy functions, written and maintained by Merlin, Inc. \\ 
 \\ 
(unless (find-package "ALCHEMY") \\ 
~~(load "alchemy-package")) \\ 
 \\ 
(in-package 'alchemy) \\ 
 \\ 
(defun lead-to-gold (x) \\ 
~~"Takes a quantity of lead and returns gold." \\ 
~~(when (> (phlogiston:heat-flow 5 x x)~~;Using a qualified symbol \\ 
~~~~~~~~~~~3) \\ 
~~~~(make-fire-bottle x))~~~~~~~~~~~~~~~~;Using an imported symbol \\ 
~~(gild x)) \\ 
 \\ 
;;; And so on ...
\end{lisp}
\end{table}
}

{\newpage
\clearpage
\samepage \begin{table}% latex2html id marker 11260
[t]
\caption{File \protect\cd{phlogiston-package} Using \protect\cd{defpackage}}
\label{DEFPACKAGE-PHLOGISTON-PACKAGE-TABLE}
\begin{lisp}
;;;; Phlogiston package definition, by Thermofluidics, Ltd. \\ 
 \\ 
;;; This package uses functions from the ALCHEMY package. \\ 
 \\ 
(cl:unless (cl:find-package "ALCHEMY") \\ 
~~(cl:load "alchemy-package")) \\ 
 \\ 
(cl:defpackage "PHLOGISTON" \\ 
~~(:use "COMMON-LISP" "ALCHEMY") \\ 
~~(:export "HEAT-FLOW" \\ 
~~~~~~~~~~~"COLD-FLOW" \\ 
~~~~~~~~~~~"MIX-FLUIDS" \\ 
~~~~~~~~~~~"SEPARATE-FLUIDS" \\ 
~~~~~~~~~~~"BURN" \\ 
~~~~~~~~~~~"MAKE-FIRE-BOTTLE") \\ 
~~)
\end{lisp}
\end{table}
}

{\newpage
\clearpage
\samepage \begin{table}% latex2html id marker 11267
[b]
\caption{File \protect\cd{phlogiston} Using \protect\cd{defpackage}}
\label{DEFPACKAGE-PHLOGISTON-FILE-TABLE}
\begin{lisp}
;;;; Phlogiston functions, by Thermofluidics, Ltd. \\ 
 \\ 
(unless (find-package "PHLOGISTON") \\ 
~~(load "phlogiston-package")) \\ 
 \\ 
(in-package 'phlogiston) \\ 
\\ 
(defvar *feeling-weak* nil) \\ 
\\ 
(defun heat-flow (amount x y) \\ 
~~"Make some amount of heat flow from x to y." \\ 
~~(when *feeling-weak* \\ 
~~~~(quaff (elixir-of-life)))~~~~~;No qualifier is needed. \\ 
~~(push-heat amount x y)) \\ 
 \\ 
;;; And so on ...
\end{lisp}
\end{table}
}

\stepcounter{chapter}
\stepcounter{section}
\stepcounter{section}
\stepcounter{section}
\stepcounter{section}
\stepcounter{section}
\stepcounter{subsection}
{\newpage
\clearpage
\samepage \( {\it x}=\mathop{\rm cis}\nolimits {2 \pi \over 3}= -{1 \over 2} + {\sqrt{3} \over 2}{\it i} \)
}

{\newpage
\clearpage
\samepage \( \sqrt{{\it x}^3} = \sqrt{1} = 1 \)
}

{\newpage
\clearpage
\samepage \( {\it x}^{3 / 2} = {\it e}^{(3/2) \log {\hbox{\scriptsize\it x}}}
   = {\it e}^{(3/2) (2\pi/3) \hbox{\scriptsize\it i}} = {\it e}^{\pi \hbox{\scriptsize\it i}} = -1 \)
}

{\newpage
\clearpage
\samepage \( \sqrt{{\it x}^3} = \sqrt{-1} = {\it i} \)
}

{\newpage
\clearpage
\samepage \( {\it x}^{3/2} = {\it e}^{(3 / 2) \log {\hbox{\scriptsize\it x}}}
   = {\it e}^{(3/2) \pi \hbox{\scriptsize\it \hbox{\scriptsize\it i}}} = -{\it i} \)
}

{\newpage
\clearpage
\samepage \( \log \left| {\it z} \right| + {\it i} \mathop{\rm phase}\nolimits {\it z} \)
}

{\newpage
\clearpage
\samepage \( {\it e}^{(\log \hbox{\scriptsize\it z})/2} \)
}

\stepcounter{subsection}
{\newpage
\clearpage
\samepage \( {\it e} ^ {\hbox{\scriptsize\it i} \cdot \hbox{\scriptsize\it radians}} \)
}

{\newpage
\clearpage
\samepage % latex2html id marker 40832
\( {\it e} ^{ \hbox{\scriptsize\it i} \theta } = \cos \theta + {\it i} \sin \theta \)
}

{\newpage
\clearpage
\samepage \( -{\it i} \log \left({\it i}{\it z} + \sqrt{1-{\it z}^2}\right) \)
}

{\newpage
\clearpage
\samepage \( -{\it i} \log \left({\it z} + {\it i}\sqrt{1-{\it z}^2}\right) \)
}

{\newpage
\clearpage
\samepage \(  \displaystyle { 2 \log \left( \sqrt{{1+{\it z} \over 2}} + {\it i} \sqrt{{1-{\it z} \over 2}} \right) \over i}\)
}

{\newpage
\clearpage
\samepage \( (\pi/2)-\arcsin {\it z} \)
}

{\newpage
\clearpage
\samepage \begin{tabular*}{\linewidth}{@{}l@{\extracolsep{\fill}}llc@{}}
\multicolumn{2}{c}{Condition}&Cartesian Locus&Range of Result \\ 
\hlinesp
\hbox to 0.4in{${\it y}=0$\hss}&\hbox to 0.4in{${\it x}>0$\hss}&\hbox to 1.6in{Positive {\it x}-axis\hss}&$0$ \\ 
${\it y}>0$&${\it x}>0$&Quadrant I&$0 < {\rm result} < \pi/2$ \\ 
${\it y}>0$&${\it x}=0$&Positive {\it y}-axis&$\pi/2$ \\ 
${\it y}>0$&${\it x}<0$&Quadrant II&$\pi/2 < {\rm result} < \pi$ \\ 
${\it y}=0$&${\it x}<0$&Negative {\it x}-axis&$\pi$ \\ 
${\it y}<0$&${\it x}<0$&Quadrant III&$-\pi < {\rm result} < -\pi/2$ \\ 
${\it y}<0$&${\it x}=0$&Negative {\it y}-axis&$-\pi/2$ \\ 
${\it y}<0$&${\it x}>0$&Quadrant IV&$-\pi/2 < {\rm result} < 0$ \\ 
${\it y}=0$&${\it x}=0$&Origin&error \\ 
\hline
\end{tabular*}
}

{\newpage
\clearpage
\samepage \begin{tabular*}{\linewidth}{@{}l@{\extracolsep{\fill}}llc@{}}
\multicolumn{2}{c}{Condition}&Cartesian Locus&Range of Result \\ 
\hlinesp
\hbox to 0.4in{${\it y}=+0$\hss}&\hbox to 0.4in{${\it x}>0$\hss}&\hbox to 1.6in{Just above positive {\it x}-axis\hss}&$+0$ \\ 
${\it y}>0$&${\it x}>0$&Quadrant I&$+0 < {\rm result} < \pi/2$ \\ 
${\it y}>0$&${\it x}=\pm 0$&Positive {\it y}-axis&$\pi/2$ \\ 
${\it y}>0$&${\it x}<0$&Quadrant II&$\pi/2 < {\rm result} < \pi$ \\ 
${\it y}=+0$&${\it x}<0$&Just below negative {\it x}-axis&$\pi$ \\ 
${\it y}=-0$&${\it x}<0$&Just above negative {\it x}-axis&$\pi$ \\ 
${\it y}<0$&${\it x}<0$&Quadrant III&$-\pi < {\rm result} < -\pi/2$ \\ 
${\it y}<0$&${\it x}=\pm 0$&Negative {\it y}-axis&$-\pi/2$ \\ 
${\it y}<0$&${\it x}>0$&Quadrant IV&$-\pi/2 < {\rm result} < -0$ \\ 
${\it y}=-0$&${\it x}>0$&Just below positive {\it x}-axis&$-0$ \\ 
${\it y}=+0$&${\it x}=+0$&Near origin&$+0$ \\ 
${\it y}=-0$&${\it x}=+0$&Near origin&$-0$ \\ 
${\it y}=+0$&${\it x}=-0$&Near origin&$\pi$ \\ 
${\it y}=-0$&${\it x}=-0$&Near origin&$-\pi$ \\ 
\hline
\end{tabular*}
}

{\newpage
\clearpage
\samepage \( -{\it i}\log \left((1+{\it i}{\it y}) \sqrt{1/(1+{\it y}^2)}\right) \)
}

{\newpage
\clearpage
\samepage \( \displaystyle { \log (1+{\it i}{\it y}) - \log (1-{\it i}{\it y}) \over 2{\it i} } \)
}

{\newpage
\clearpage
\samepage $-\pi/2$
}

{\newpage
\clearpage
\samepage $\pi/2$
}

{\newpage
\clearpage
\samepage \( ({\it e}^{\hbox{\scriptsize\it z}}-{\it e}^{-\hbox{\scriptsize\it z}})/2 \)
}

{\newpage
\clearpage
\samepage \( ({\it e}^{\hbox{\scriptsize\it z}}+{\it e}^{-\hbox{\scriptsize\it z}})/2 \)
}

{\newpage
\clearpage
\samepage \( ({\it e}^{\hbox{\scriptsize\it z}}-{\it e}^{-\hbox{\scriptsize\it z}})/({\it e}^{\hbox{\scriptsize\it z}}+{\it e}^{-\hbox{\scriptsize\it z}}) \)
}

{\newpage
\clearpage
\samepage \( \log \left({\it z}+\sqrt{1+{\it z}^2}\right) \)
}

{\newpage
\clearpage
\samepage \( \log \left({\it z}+({\it z}+1)\sqrt{({\it z}-1)/({\it z}+1)}\right) \)
}

{\newpage
\clearpage
\samepage \( \log \left((1+{\it z})\sqrt{1-1/{\it z}^2}\right) \)
}

{\newpage
\clearpage
\samepage \( \log \left((1+{\it z})\sqrt{1/(1-{\it z}^2)}\right) \)
}

\stepcounter{subsection}
{\newpage
\clearpage
\samepage $(-\pi,\pi]$
}

{\newpage
\clearpage
\samepage $[-\pi,\pi]$
}

{\newpage
\clearpage
\samepage \( \sqrt{{\it z}} = {\it e}^{(\log \hbox{\scriptsize\it z})/2} \)
}

{\newpage
\clearpage
\samepage \( \mathop{\rm phase}\nolimits {\it z} = \arctan (\Im {\it z}, \Re {\it z}) \)
}

{\newpage
\clearpage
\samepage \( \log {\it z} = \left(\log \left|{\it z}\right|\right)+{\it i} (\mathop{\rm phase}\nolimits {\it z}) \)
}

{\newpage
\clearpage
\samepage \( \log_{\hbox{\scriptsize\it b}} {\it z}=(\log {\it z})/(\log {\it b}) \)
}

{\newpage
\clearpage
\samepage \( {\it b}^{\hbox{\scriptsize\it x}}={\it e}^{\hbox{\scriptsize\it x} \log \hbox{\scriptsize\it b} } \)
}

{\newpage
\clearpage
\samepage $0^0=1$
}

{\newpage
\clearpage
\samepage ${\it b}^{\hbox{\scriptsize\it x}}=0$
}

{\newpage
\clearpage
\samepage $0^{\hbox{\scriptsize\it x}}$
}

{\newpage
\clearpage
\samepage \( \arcsin {\it z}=-{\it i} \log \left({\it i}{\it z}+\sqrt{1-{\it z}^2}\right) \)
}

{\newpage
\clearpage
\samepage \( \displaystyle \arcsin {\it z} = { \mathop{\rm arcsinh}\nolimits {\it i}{\it z} \over {\it i} }\)
}

{\newpage
\clearpage
\samepage \( \arccos {\it z}=-{\it i} \log \left({\it z}+{\it i} \sqrt{1-{\it z}^2}\right) \)
}

{\newpage
\clearpage
\samepage \( \arccos {\it z}={\pi \over 2}-\arcsin {\it z} \)
}

{\newpage
\clearpage
\samepage \( \arctan {\it z}=-{\it i} \log \left((1+{\it i} {\it z}) \sqrt{1/(1+{\it z}^2)}\right)\)
}

{\newpage
\clearpage
\samepage \( \displaystyle \arctan {\it z} = { \log (1+{\it i}{\it z}) - \log (1-{\it i}{\it z}) \over 2{\it i} }\)
}

{\newpage
\clearpage
\samepage \( \displaystyle \arctan {\it z} = { \mathop{\rm arctanh}\nolimits {\it i}{\it z} \over {\it i} }\)
}

{\newpage
\clearpage
\samepage \( \mathop{\rm arcsinh}\nolimits {\it z}=\log \left({\it z}+\sqrt{1+{\it z}^2}\right)\)
}

{\newpage
\clearpage
\samepage \( \mathop{\rm arccosh}\nolimits {\it z}=\log \left({\it z}+({\it z}+1)\sqrt{({\it z}-1)/({\it z}+1)}\right) \)
}

{\newpage
\clearpage
\samepage \( \mathop{\rm arccosh}\nolimits {\it z}=2 \log \left(  \sqrt{({\it z}+1)/2} + \sqrt{({\it z}-1)/2} \right) \)
}

{\newpage
\clearpage
\samepage \( \mathop{\rm arctanh}\nolimits {\it z}=\log \left((1+{\it z})\sqrt{1-1/{\it z}^2}\right) \)
}

{\newpage
\clearpage
\samepage \( \mathop{\rm arctanh}\nolimits {\it z}=\log \left((1+{\it z})\sqrt{1/(1-{\it z}^2)}\right) \)
}

{\newpage
\clearpage
\samepage $-\pi {\it i}/2$
}

{\newpage
\clearpage
\samepage $\pi {\it i}/2$
}

{\newpage
\clearpage
\samepage \( \displaystyle \mathop{\rm arctanh}\nolimits {\it z}= { \left(\log(1+{\it z}) - \log(1-{\it z})\right) \over 2 } \)
}

{\newpage
\clearpage
\samepage \begin{tabular*}{\textwidth}{@{}l@{\extracolsep{\fill}}ll@{}}
$\sin {\it i}{\it z} = {\it i} \sinh {\it z}$&$\sinh {\it i}{\it z} = {\it i} \sin {\it z}$&$\arctan {\it i}{\it z} = {\it i} \mathop{\rm arctanh}\nolimits {\it z}$ \\ 
$\cos {\it i}{\it z} = \cosh {\it z}$&$\cosh {\it i}{\it z} = \cos {\it z}$&$\mathop{\rm arcsinh}\nolimits {\it i}{\it z} = {\it i} \arcsin {\it z}$ \\ 
$\tan {\it i}{\it z} = {\it i} \tanh {\it z}$&$\arcsin {\it i}{\it z} = {\it i} \mathop{\rm arcsinh}\nolimits {\it z}$&$\mathop{\rm arctanh}\nolimits {\it i}{\it z} = {\it i} \arctan {\it z}$
\end{tabular*}
}

{\newpage
\clearpage
\samepage $[1/4,1/2]$
}

{\newpage
\clearpage
\samepage $[3/4, 1]$
}

{\newpage
\clearpage
\samepage $[\pi /2, 2]$
}

{\newpage
\clearpage
\samepage $[3, \pi]$
}

{\newpage
\clearpage
\samepage $2^{\hbox{\scriptsize\it j}+1}$
}

{\newpage
\clearpage
\samepage ${\it f}({\it z})$
}

{\newpage
\clearpage
\samepage $[-4.1, 4.1]$
}

{\newpage
\clearpage
\samepage $\sqrt{1-{\it z}^2}$
}

{\newpage
\clearpage
\samepage $\sqrt{1+{\it z}^2}$
}

{\newpage
\clearpage
\samepage $({\it z}-1)/({\it z}+1)$
}

{\newpage
\clearpage
\samepage $(1+{\it z})/(1-{\it z})$
}

{\newpage
\clearpage
\samepage ${\it f}({\it z})=({\it z}-1)/({\it z}+1)$
}

{\newpage
\clearpage
\samepage $\tanh {\it z} = {\it f}({\it e}^{2\hbox{\scriptsize\it z}})$
}

{\newpage
\clearpage
\samepage ${\it g}({\it z})=\sqrt{1-{\it z}^2}$
}

{\newpage
\clearpage
\samepage $\cos {\it z} = {\it g}(\sin {\it z})$
}

{\newpage
\clearpage
\samepage \( \displaystyle {\it z}=\lim_{\hbox{\scriptsize\it x}\rightarrow +\infty} {\it f}({\it x}+0{\it i\/}) \)
}

{\newpage
\clearpage
\samepage \( \displaystyle {\it z} = \lim_{\hbox{\scriptsize\it y}\rightarrow -\infty} {\it f}(0+{\it y}{\it i\/}) \)
}

{\newpage
\clearpage
\samepage \begin{figure}% latex2html id marker 12561
\vbox to \foodimen{
\caption{Initial Decoration of the Complex Plane (Identity Function)}
\label{IDENTITY-PLOT}
\vskip -2pt \vskip 28pc \PostScriptFile{identity-plot.ps}\vskip 3pt
\vss
\small\noindent
This figure was produced in exactly the same manner as succeeding figures,
simply by plotting the function \cd{identity} instead of a numerical function.
Thus the first of these figures was produced by the last function of the first edition.
I knew it would come in handy someday!
}
\end{figure}
}

{\newpage
\clearpage
\samepage \begin{figure}% latex2html id marker 12567
\vbox to \foodimen{

\caption{Illustration of the Range of the Square Root Function}
\label{SECOND-PLOT}
\vskip -2pt \vskip 28pc \PostScriptFile{sqrt-plot.ps}\vskip 3pt
\vss
\small\noindent
The \cd{sqrt} function maps the complex plane into the right half
of the plane by slitting it along the negative real axis and then sweeping
it around as if half-closing a folding fan.
The fan also shrinks, as if it were made of cotton and had gotten
wetter at the periphery than at the center.
The positive real axis is mapped onto itself.  The negative real axis is mapped
onto the positive imaginary axis (but if minus zero is supported, then
$-{\it x}+0{\it i}$ is mapped onto the positive imaginary axis and $-{\it x}-0{\it i}$ onto
the negative imaginary axis, assuming ${\it x}>0$).  The positive imaginary axis
is mapped onto the northeast diagonal, and the negative imaginary axis
onto the southeast diagonal.  More generally, lines are mapped to rectangular hyperbolas
(or fragments thereof\,) centered on the origin;
lines through the origin are mapped to degenerate
hyperbolas (perpendicular lines through the origin).
}
\end{figure}
}

{\newpage
\clearpage
\samepage \begin{figure}% latex2html id marker 12578
\vbox to \foodimen{
\caption{Illustration of the Range of the Exponential Function}
\vskip -2pt \vskip 28pc \PostScriptFile{exp-plot.ps}\vskip 3pt
\vss
\small\noindent
The \cd{exp} function maps horizontal lines to radii and maps vertical
lines to circles centered at the origin.
The origin is mapped to 1.  (It is instructive to compare
this graph with those of other functions
that map the origin to 1, for example $(1+{\it z})/(1-{\it z})$, $\cos {\it z}$, and $\sqrt{1-{\it z}^2}$.)
The entire real axis is mapped to the positive
real axis, with $-\infty$ mapping to the origin and $+\infty$ to itself.
The imaginary axis is mapped to the unit circle with infinite multiplicity (period $2\pi$);
therefore the mapping of the imaginary infinities $\pm\infty {\it i}$ is indeterminate.
It follows that the entire left half-plane is mapped to the interior of the unit circle,
and the right half-plane is mapped to the exterior of the unit circle.
A line at any angle other than horizontal or vertical is mapped to a
logarithmic spiral (but this is not illustrated here).
}
\end{figure}
}

{\newpage
\clearpage
\samepage \begin{figure}% latex2html id marker 12588
\vbox to \foodimen{
\caption{Illustration of the Range of the Natural Logarithm Function}
\vskip -2pt \vskip 28pc \PostScriptFile{log-plot.ps}\vskip 3pt
\vss
\small\noindent
The \cd{log} function, which is the inverse of \cd{exp}, naturally maps radial lines to
horizontal lines and circles centered at the origin to vertical lines.
The interior of the unit circle is thus mapped to the entire left half-plane,
and the exterior of the unit circle is mapped to the right half-plane.
The positive real axis is mapped to the entire real axis, and the negative
real axis to a horizontal line of height $\pi$.  The positive and negative
imaginary axes are mapped to horizontal lines of height $\pm\pi/2$.
The origin is mapped to $-\infty$.

}
\end{figure}
}

{\newpage
\clearpage
\samepage \begin{figure}% latex2html id marker 12594
\vbox to \foodimen{
\caption{Illustration of the Range of the Function $({\it z}-1)/({\it z}+1)$}
\vskip -2pt \vskip 28pc \PostScriptFile{minus-over-plus-plot.ps}\vskip 3pt
\vss
\small\noindent
A line is a degenerate circle with infinite radius;
when I say ``circles'' here I also mean lines.
Then $({\it z}-1)/({\it z}+1)$ maps circles into circles.
All circles through $-1$ become lines; all lines become
circles through $1$.
The real axis is mapped onto itself: 1 to
the origin, the origin to $-1$, $-1$ to infinity, and infinity to 1.
The imaginary axis becomes the unit circle; {\it i} is mapped to itself,
as is $-{\it i}$.  Thus the entire right half-plane is mapped to the interior
of the unit circle, the unit circle interior to the left half-plane,
the left half-plane to the unit circle exterior, and the unit circle exterior
to the right half-plane.  Imagine the complex plane to be a vast sea.
The Colossus of Rhodes straddles the origin, its left foot on {\it i} and its right foot on $-{\it i}$.
It bends down and briefly paddles water between its legs so furiously that the water
directly beneath is pushed out into the entire area behind it; much that was
behind swirls forward to either side; and all that was before is
sucked in to lie between its feet.
}
\end{figure}
}

{\newpage
\clearpage
\samepage \begin{figure}% latex2html id marker 12605
\vbox to \foodimen{
\caption{Illustration of the Range of the Function $(1+{\it z})/(1-{\it z})$}
\vskip -2pt \vskip 28pc \PostScriptFile{plus-over-minus-plot.ps}\vskip 3pt
\vss
\small\noindent
The function ${\it h}({\it z})=(1+{\it z})/(1-{\it z})$
is the inverse of ${\it f}({\it z})=({\it z}-1)/({\it z}+1)$; that is,
${\it h}({\it f}({\it z}))={\it f}({\it h}({\it z}))={\it z}$. At first glance,
the graph of ${\it h}$ appears to be that of ${\it f}$
flipped left-to-right, or perhaps reflected in the origin, but careful
consideration of the shaded annuli reveals that this is not so; something more
subtle is going on.  Note that
${\it f}({\it f}({\it z}))={\it h}({\it h}({\it z}))={\it g}({\it z})=-1/{\it z}$.
The functions ${\it f}$, ${\it g}$, ${\it h}$, and the identity function
thus form a group under composition, isomorphic to the group
of the cyclic permutations of the points $-1$, $0$, $1$, and $\infty$, as
indeed these functions accomplish the four possible cyclic permutations
on those points.  This function group is a subset of the group of bilinear
transformations $({\it a}{\it z}+{\it b})/({\it c}{\it z}+{\it d})$,
all of which are conformal (angle-preserving) and map circles
onto circles.  Now, doesn't that tangle of circles through $-1$ look like something
the cat got into?
}
\end{figure}
}

{\newpage
\clearpage
\samepage \begin{figure}% latex2html id marker 12645
\vbox to \foodimen{
\caption{Illustration of the Range of the Sine Function}
\vskip -2pt \vskip 28pc \PostScriptFile{sin-plot.ps}\vskip 3pt
\vss
\small\noindent
We are used to seeing \cd{sin} looking like a wiggly ocean wave,
graphed vertically as a function of the real axis only.  Here is a different view.
The entire real axis is mapped to the segment $[-1, 1]$ of the real axis
with infinite multiplicity (period $2\pi$).  The imaginary axis is mapped to itself
as if by \cd{sinh} considered as a real function.  The origin is mapped to itself.
Horizontal lines are mapped to ellipses with foci at $\pm1$ (note that two horizontal
lines equidistant from the real axis will map onto the same ellipse).
Vertical lines are mapped to hyperbolas with the same foci.  There is a curious accident:
the ellipse for horizontal
lines at distance $\pm1$ from the real axis appears to intercept the real axis at
$\pm\pi/2\approx \pm1.57\ldots$ but this is not so; the intercepts are actually at
$\pm({\it e}+1/{\it e})/2\approx \pm1.54\ldots\,\hbox{}$.
}
\end{figure}
}

{\newpage
\clearpage
\samepage \begin{figure}% latex2html id marker 12654
\vbox to \foodimen{
\caption{Illustration of the Range of the Arc Sine Function}
\vskip -2pt \vskip 28pc \PostScriptFile{asin-plot.ps}\vskip 3pt
\vss
\small\noindent
Just as \cd{sin} grabs horizontal lines and bends them into elliptical loops around
the origin, so its inverse \cd{asin} takes annuli and yanks them more
or less horizontally straight.  Because sine is not injective,
its inverse as a function cannot be surjective.  This is just a highfalutin
way of saying that the range of the \cd{asin} function doesn't
cover the entire plane but only a strip $\pi$ wide; arc sine as a one-to-many relation
would cover the plane with an infinite number of copies of this strip side by side,
looking for all the world like the tail of a peacock with an infinite number of feathers.
The imaginary axis is mapped to itself as if by \cd{asinh} considered as a real function.
The real axis is mapped to a bent path, turning corners at $\pm\pi/2$ (the points to
which $\pm 1$ are mapped); $+\infty$ is mapped to $\pi/2-\infty {\it i}$, and
$-\infty$ to $-\pi/2+\infty {\it i}$.
}
\end{figure}
}

{\newpage
\clearpage
\samepage \begin{figure}% latex2html id marker 12664
\vbox to \foodimen{
\caption{Illustration of the Range of the Cosine Function}
\vskip -2pt \vskip 28pc \PostScriptFile{cos-plot.ps}\vskip 3pt
\vss
\small\noindent
We are used to seeing \cd{cos} looking exactly like \cd{sin}, a wiggly ocean wave,
only displaced.  Indeed the complex mapping of \cd{cos} is also similar
to that of \cd{sin}, with horizontal and vertical lines mapping to the same ellipses
and hyperbolas with foci at $\pm 1$, although mapping to them in a different
manner, to be sure.
The entire real axis is again mapped to the segment $[-1, 1]$ of the real axis,
but each half of the imaginary axis is mapped to the real axis to the right of 1
(as if by \cd{cosh} considered as a real function).  Therefore $\pm\infty {\it i}$
both map to $+\infty$.
The origin is mapped to 1.  Whereas \cd{sin} is an odd function, \cd{cos} is an
even function; as a result {\it two} points in each annulus, one the negative
of the other, are mapped to the same shaded point in this graph; the shading shown here
is taken from points in the original upper half-plane.
}
\end{figure}
}

{\newpage
\clearpage
\samepage \begin{figure}% latex2html id marker 12677
\vbox to \foodimen{
\caption{Illustration of the Range of the Arc Cosine Function}
\vskip -2pt \vskip 28pc \PostScriptFile{acos-plot.ps}\vskip 3pt
\vss
\small\noindent
The graph of \cd{acos} is very much like that of \cd{asin}.
One might think that our nervous peacock has shuffled half a step
to the right, but the shading on the annuli shows that we have instead caught
the bird exactly in mid-flight while doing a cartwheel.
This is easily understood if we recall that $\arccos {\it z}=(\pi/2)-\arcsin {\it z}$;
negating $\arcsin {\it z}$ rotates it upside down, and adding the result to $\pi/2$
translates it $\pi/2$ to the right.
The imaginary axis is mapped upside down to the vertical line at $\pi/2$.
The point $+1$ is mapped to the origin, and $-1$ to $\pi$.
The image of the real axis is again cranky; $+\infty$ is mapped to $+\infty {\it i}$,
and $-\infty$ to $\pi-\infty {\it i}$.

}
\end{figure}
}

{\newpage
\clearpage
\samepage \begin{figure}% latex2html id marker 12688
\vbox to \foodimen{
\caption{Illustration of the Range of the Tangent Function}
\vskip -2pt \vskip 28pc \PostScriptFile{tan-plot.ps}\vskip 3pt
\vss
\small\noindent
The usual graph of \cd{tan} as a real function looks like an infinite chorus
line of disco dancers, left hands pointed skyward and right hands to the floor.
The \cd{tan} function is the quotient of \cd{sin} and \cd{cos}
but it doesn't much look like either except for having period $2\pi$.
This goes for the complex plane as well, although the swoopy loops produced
from the annulus between $\pi/2$ and $2$ look vaguely like
those from the graph of \cd{sin} inside out.
The real axis is mapped onto itself with infinite multiplicity (period $2\pi$).
The imaginary axis is mapped backwards onto $[-{\it i},{\it i}]$:
$+\infty {\it i}$ is mapped to $-{\it i}$ and $-\infty {\it i}$ to $+{\it i}$.
Horizontal lines below or above the real axis
become circles surrounding $+{\it i}$ or $-{\it i}$, respectively.
Vertical lines become circular arcs from $+{\it i}$ to $-{\it i}$;
two vertical lines separated by $(2{\it k}+1)\pi$ for integer ${\it k}$
together become a complete circle.  It seems that two arcs
shown hit the real axis at $\pm\pi/2=\pm 1.57\ldots$ but that is a coincidence;
they really hit the axis at $\pm\tan 1= 1.55\ldots\,\hbox{}$.
}
\end{figure}
}

{\newpage
\clearpage
\samepage \begin{figure}% latex2html id marker 12710
\vbox to \foodimen{
\caption{Illustration of the Range of the Arc Tangent Function}
\vskip -2pt \vskip 28pc \PostScriptFile{xatan-plot.ps}\vskip 3pt
\vss
\small\noindent
All I can say is that this peacock is a horse of another color.
At first glance, the axes seem to map in the same way as for \cd{asin} and
\cd{acos}, but look again: this time it's the imaginary axis doing weird things.
All infinities map multiply to the points
$(2{\it k}+1)\pi/2$; within the strip of principal values we may say that
the real axis is mapped to the interval $[-\pi/2,+\pi/2]$ and therefore
$-\infty$ is mapped to $-\pi/2$ and $+\infty$ to $+\pi/2$.
The point $+{\it i}$ is mapped to $+\infty {\it i}$, and $-{\it i}$ to $-\infty {\it i}$, and
so the imaginary axis is mapped into three pieces: the segment
$[-\infty {\it i},-{\it i}]$ is mapped to $[\pi/2,\pi/2-\infty {\it i}]$; the segment
$[-{\it i},{\it i}]$ is mapped to the imaginary axis $[-\infty {\it i},+\infty {\it i}]$; and the segment
$[+{\it i},+\infty {\it i}]$ is mapped to $[-\pi/2+\infty {\it i},-\pi/2]$.
}
\end{figure}
}

{\newpage
\clearpage
\samepage \begin{figure}% latex2html id marker 12731
\vbox to \foodimen{
\caption{Illustration of the Range of the Hyperbolic Sine Function}
\vskip -2pt \vskip 28pc \PostScriptFile{sinh-plot.ps}\vskip 3pt
\vss
\small\noindent
It would seem that the graph of \cd{sinh} is merely that of \cd{sin}
rotated 90 degrees.  If that were so, then we would have $\sinh {\it z} = {\it i} \sin {\it z}$.
Careful inspection of the shading, however, reveals that this is not quite the case;
in both graphs the lightest and darkest shades, which initially are adjacent to
the positive real axis, remain adjacent to the positive real axis in both cases.
To derive the graph of \cd{sinh} from \cd{sin} we must therefore first
rotate the complex plane by $-90$ degrees, then apply \cd{sin}, then
rotate the result by 90 degrees. In other words, $\sinh {\it z} = {\it i} \sin (-{\it i\/}){\it z}$;
consistently replacing ${\it z}$ with ${\it i}{\it z}$ in this formula yields the familiar identity
$\sinh {\it i}{\it z} = {\it i} \sin {\it z}$.
}
\end{figure}
}

{\newpage
\clearpage
\samepage \begin{figure}% latex2html id marker 12754
\vbox to \foodimen{
\caption{Illustration of the Range of the Hyperbolic Arc Sine Function}
\vskip -2pt \vskip 28pc \PostScriptFile{asinh-plot.ps}\vskip 3pt
\vss
\small\noindent
The peacock sleeps.  Because $\mathop{\rm arcsinh}\nolimits {\it i}{\it z} = {\it i} \arcsin {\it z}$,
the graph of \cd{asinh}
is related to that of \cd{asin} by pre- and post-rotations of the complex plane
in the same way as for \cd{sinh} and \cd{sin}.
}
\end{figure}
}

{\newpage
\clearpage
\samepage \begin{figure}% latex2html id marker 12766
\vbox to \foodimen{
\caption{Illustration of the Range of the Hyperbolic Cosine Function}
\vskip -2pt \vskip 28pc \PostScriptFile{cosh-plot.ps}\vskip 3pt
\vss
\small\noindent
The graph of \cd{cosh} does {\it not} look like that of \cd{cos} rotated
90 degrees; instead it looks like that of \cd{cos} unrotated.
That is because $\cosh {\it i}{\it z}$ is not equal to ${\it i} \cos {\it z}$;
rather, $\cosh {\it i}{\it z} = \cos {\it z}$.
Interpreted, that means that the shading is pre-rotated but there is no
post-rotation.
}
\end{figure}
}

{\newpage
\clearpage
\samepage \begin{figure}% latex2html id marker 12781
\vbox to \foodimen{
\caption{Illustration of the Range of the Hyperbolic Arc Cosine Function}
\vskip -2pt \vskip 28pc \PostScriptFile{acosh-plot.ps}\vskip 3pt
\vss
\small\noindent
Hmm---I'd rather not say what happened to this peacock.
This feather looks a bit mangled.  Actually it is all right---the principal
value for \cd{acosh} is so chosen that its graph does not look simply
like a rotated version of the graph of \cd{acos}, but if all values were
shown, the two graphs would fill the plane in repeating patterns related
by a rotation.

}
\end{figure}
}

{\newpage
\clearpage
\samepage \begin{figure}% latex2html id marker 12787
\vbox to \foodimen{
\caption{Illustration of the Range of the Hyperbolic Tangent Function}
\vskip -2pt \vskip 28pc \PostScriptFile{tanh-plot.ps}\vskip 3pt
\vss
\small\noindent
The diagram for \cd{tanh} is simply that of \cd{tan} turned on its ear:
${\it i} \tan {\it z} = \tanh {\it i}{\it z}$.
The imaginary axis is mapped onto itself with infinite multiplicity (period $2\pi$),
and the real axis is mapped onto the segment $[-1,+1]$:
$+\infty$ is mapped to $+1$, and $-\infty$ to $-1$.
Vertical lines to the left or right of the real axis
are mapped to circles surrounding $-1$ or $1$, respectively.
Horizontal lines are mapped to circular arcs anchored at $-1$ and $+1$;
two horizontal lines separated by a distance $(2{\it k}+1)\pi$ for integer ${\it k}$ are
together mapped into a complete circle.  How do we know these really are circles?
Well, $\tanh {\it z} = ((\exp 2{\it z})-1)/((\exp 2{\it z})+1)$, which is the composition of the
bilinear transform $({\it z}-1)/({\it z}+1)$, the exponential $\exp {\it z}$,
and the magnification $2{\it z}$.
Magnification maps lines to lines of the same slope; the exponential maps
horizontal lines to circles and vertical lines to radial lines;
and a bilinear transform maps generalized circles (including lines) to
generalized circles.  Q.E.D.
}
\end{figure}
}

{\newpage
\clearpage
\samepage \begin{figure}% latex2html id marker 12806
\vbox to \foodimen{
\caption{Illustration of the Range of the Hyperbolic Arc Tangent Function}
\label{ATANH-PLOT}
\vskip -2pt \vskip 28pc \PostScriptFile{really-good-atanh-plot.ps}\vskip 3pt
\vss
\small\noindent
A sleeping peacock of another color: $\mathop{\rm arctanh}\nolimits {\it i}{\it z} = {\it i} \arctan {\it z}$.
}
\end{figure}
}

{\newpage
\clearpage
\samepage \begin{figure}% latex2html id marker 12815
\vbox to \foodimen{
\caption{Illustration of the Range of the Function $\protect\sqrt{1-{\it z}^2}$}
\vskip -2pt \vskip 28pc \PostScriptFile{sqrt-one-minus-sq-plot.ps}\vskip 3pt
\vss
\small\noindent
Here is a curious graph indeed for so simple a function!
The origin is mapped to 1.  The real axis segment $[0,1]$ is
mapped backwards (and non-linearly) into itself; the segment $[1,+\infty]$
is mapped non-linearly onto the positive imaginary axis.
The negative real axis is mapped to the same points as the positive real axis.
Both halves of the imaginary axis are mapped into $[1,+\infty]$ on the real axis.
Horizontal lines become vaguely vertical, and
vertical lines become vaguely horizontal.
Circles centered at the origin are transformed into Cassinian \hbox{(half-)ovals}; the unit
circle is mapped to a \hbox{(half-)lemniscate} of Bernoulli.  The outermost annulus appears
to have its {\it inner} edge at $\pi$ on the real axis and its {\it outer}
edge at 3 on the imaginary axis, but this is another accident; the intercept
on the real axis, for example, is not really at $\pi\approx 3.14\ldots$ but
at $\sqrt{1-(3{\it i\/})^2}=\sqrt{10}\approx 3.16\ldots\,\hbox{}$.

}
\end{figure}
}

{\newpage
\clearpage
\samepage \begin{figure}% latex2html id marker 12826
\vbox to \foodimen{
\caption{Illustration of the Range of the Function $\protect\sqrt{1+{\it z}^2}$}
\label{LAST-PLOT}
\vskip -2pt \vskip 28pc \PostScriptFile{sqrt-one-plus-sq-plot.ps}\vskip 3pt
\vss
\small\noindent
The graph of ${\it q}({\it z})=\sqrt{1+{\it z}^2}$ looks like that
of ${\it p}({\it z})=\sqrt{1-{\it z}^2}$ except for
the shading.  You might not expect ${\it p}$ and ${\it q}$ to be related in the same
way that \cd{cos} and \cd{cosh} are, but after a little reflection (or perhaps
I should say, after turning it around in one's mind) one can see that
${\it q}({\it i}{\it z})={\it p}({\it z})$.  This formula is indeed of exactly the same form as
$\cosh {\it i}{\it z} = \cos {\it z}$.  The function
$\sqrt{1+{\it z}^2}$ maps both halves of the real axis into $[1,+\infty]$ on the real axis.
The segments $[0,{\it i}]$ and $[0,-{\it i}]$ of the imaginary axis are each mapped
backwards onto segment $[0,1]$ of the real axis; $[{\it i},+\infty {\it i}]$ and
$[-{\it ,}-\infty {\it i}]$ are each mapped onto the positive imaginary axis
(but if minus zero is supported then opposite sides of the
imaginary axis map to opposite halves of the imaginary axis---for example,
$q(+0+2{\it i\/})=\sqrt{5}{\it i}$ but $q(-0+2{\it i\/})=-\sqrt{5}{\it i}$).
}
\end{figure}
}

\stepcounter{section}
{\newpage
\clearpage
\samepage $\hbox{\it integer}+\hbox{\rm 0.5}$
}

{\newpage
\clearpage
\samepage $-1.0$
}

{\newpage
\clearpage
\samepage $1.0$
}

{\newpage
\clearpage
\samepage $90^\circ$
}

{\newpage
\clearpage
\samepage \( (4+0{\it i\/})(+0+{\it i\/}) = ((4)(+0) - (+0)(1)) + ((4)(1) + (+0)(+0)){\it i} \)
}

{\newpage
\clearpage
\samepage \( = ((+0)-(+0))+((4)+(+0)){\it i} = +0+4{\it i} \)
}

\stepcounter{section}
{\newpage
\clearpage
\samepage $2^{\hbox{\scriptsize\it j}}$
}

{\newpage
\clearpage
\samepage $2^{\hbox{\scriptsize\it index}}$
}

{\newpage
\clearpage
\samepage ${\it floor}({\it integer}\cdot2^{\hbox{\scriptsize\it count}}$
}

{\newpage
\clearpage
\samepage \( {\it ceiling}(\log_2 ({\bf if}\; {\it integer} < 0 \;{\bf then} \;
    -{\it integer} \;{\bf else}\; {\it integer}+1)) \)
}

\stepcounter{section}
{\newpage
\clearpage
\samepage \( 2^{\hbox{\scriptsize\it position}+\hbox{\scriptsize\it size}-1} \)
}

{\newpage
\clearpage
\samepage \( 2^{\hbox{\scriptsize\it position}} \)
}

\stepcounter{section}
{\newpage
\clearpage
\samepage $[{\it A},{\it B})$
}

{\newpage
\clearpage
\samepage ${\it X}\cdot({\it B}-{\it A})+{\it A}$
}

{\newpage
\clearpage
\samepage $[0.0, 1.0)$
}

{\newpage
\clearpage
\samepage $[0,{\it M})$
}

{\newpage
\clearpage
\samepage ${\it X}={\it N}/{\it M}$
}

{\newpage
\clearpage
\samepage ${\it M}=2^{\hbox{\footnotesize\it f}}$
}

{\newpage
\clearpage
\samepage $[1.0, 2.0)$
}

\stepcounter{section}
{\newpage
\clearpage
\samepage $-2^{15}$
}

{\newpage
\clearpage
\samepage $2^{15}-1$
}

\stepcounter{chapter}
{\newpage
\clearpage
\samepage \begin{table}% latex2html id marker 14108
\caption{Standard Character Labels, Glyphs, and Descriptions}
\label{STANDARD-CHAR-REPERTOIRE-TABLE}
\tabcolsep0pt

\begin{tabular*}{\textwidth}{@{}l@{\extracolsep{\fill}}llllllll@{}}
           &&&\cd{SM05}&\cd{{\Xatsign}}&{\rm commercial at}&\cd{SD13}&\cd{{\Xbq}}&{\rm grave accent} \\ 
\cd{SP02}&\cd{!}&{\rm exclamation mark}&\cd{LA02}&\cd{A}&{\rm capital A}&\cd{LA01}&\cd{a}&{\rm small a} \\ 
\cd{SP04}&\cd{"}&{\rm quotation mark}&\cd{LB02}&\cd{B}&{\rm capital B}&\cd{LB01}&\cd{b}&{\rm small b} \\ 
\cd{SM01}&\cd{\#}&{\rm number sign}&\cd{LC02}&\cd{C}&{\rm capital C}&\cd{LC01}&\cd{c}&{\rm small c} \\ 
\cd{SC03}&\cd{\$}&{\rm dollar sign}&\cd{LD02}&\cd{D}&{\rm capital D}&\cd{LD01}&\cd{d}&{\rm small d} \\ 
\cd{SM02}&\cd{\%}&{\rm percent sign}&\cd{LE02}&\cd{E}&{\rm capital E}&\cd{LE01}&\cd{e}&{\rm small e} \\ 
\cd{SM03}&\cd{\&}&{\rm ampersand}&\cd{LF02}&\cd{F}&{\rm capital F}&\cd{LF01}&\cd{f}&{\rm small f} \\ 
\cd{SP05}&\cd{'}&{\rm apostrophe}&\cd{LG02}&\cd{G}&{\rm capital G}&\cd{LG01}&\cd{g}&{\rm small g} \\ 
\cd{SP06}&\cd{(}&{\rm left parenthesis}&\cd{LH02}&\cd{H}&{\rm capital H}&\cd{LH01}&\cd{h}&{\rm small h} \\ 
\cd{SP07}&\cd{)}&{\rm right parenthesis}&\cd{LI02}&\cd{I}&{\rm capital I}&\cd{LI01}&\cd{i}&{\rm small i} \\ 
\cd{SM04}&\cd{*}&{\rm asterisk}&\cd{LJ02}&\cd{J}&{\rm capital J}&\cd{LJ01}&\cd{j}&{\rm small j} \\ 
\cd{SA01}&\cd{+}&{\rm plus sign}&\cd{LK02}&\cd{K}&{\rm capital K}&\cd{LK01}&\cd{k}&{\rm small k} \\ 
\cd{SP08}&\cd{,}&{\rm comma}&\cd{LL02}&\cd{L}&{\rm capital L}&\cd{LL01}&\cd{l}&{\rm small l} \\ 
\cd{SP10}&\cd{-}&{\rm hyphen or minus sign}&\cd{LM02}&\cd{M}&{\rm capital M}&\cd{LM01}&\cd{m}&{\rm small m} \\ 
\cd{SP11}&\cd{.}&{\rm period or full stop}&\cd{LN02}&\cd{N}&{\rm capital N}&\cd{LN01}&\cd{n}&{\rm small n} \\ 
\cd{SP12}&\cd{/}&{\rm solidus}&\cd{LO02}&\cd{O}&{\rm capital O}&\cd{LO01}&\cd{o}&{\rm small o} \\ 
\cd{ND10}&\cd{0}&{\rm digit 0}&\cd{LP02}&\cd{P}&{\rm capital P}&\cd{LP01}&\cd{p}&{\rm small p} \\ 
\cd{ND01}&\cd{1}&{\rm digit 1}&\cd{LQ02}&\cd{Q}&{\rm capital Q}&\cd{LQ01}&\cd{q}&{\rm small q} \\ 
\cd{ND02}&\cd{2}&{\rm digit 2}&\cd{LR02}&\cd{R}&{\rm capital R}&\cd{LR01}&\cd{r}&{\rm small r} \\ 
\cd{ND03}&\cd{3}&{\rm digit 3}&\cd{LS02}&\cd{S}&{\rm capital S}&\cd{LS01}&\cd{s}&{\rm small s} \\ 
\cd{ND04}&\cd{4}&{\rm digit 4}&\cd{LT02}&\cd{T}&{\rm capital T}&\cd{LT01}&\cd{t}&{\rm small t} \\ 
\cd{ND05}&\cd{5}&{\rm digit 5}&\cd{LU02}&\cd{U}&{\rm capital U}&\cd{LU01}&\cd{u}&{\rm small u} \\ 
\cd{ND06}&\cd{6}&{\rm digit 6}&\cd{LV02}&\cd{V}&{\rm capital V}&\cd{LV01}&\cd{v}&{\rm small v} \\ 
\cd{ND07}&\cd{7}&{\rm digit 7}&\cd{LW02}&\cd{W}&{\rm capital W}&\cd{LW01}&\cd{w}&{\rm small w} \\ 
\cd{ND08}&\cd{8}&{\rm digit 8}&\cd{LX02}&\cd{X}&{\rm capital X}&\cd{LX01}&\cd{x}&{\rm small x} \\ 
\cd{ND09}&\cd{9}&{\rm digit 9}&\cd{LY02}&\cd{Y}&{\rm capital Y}&\cd{LY01}&\cd{y}&{\rm small y} \\ 
\cd{SP13}&\cd{:}&{\rm colon}&\cd{LZ02}&\cd{Z}&{\rm capital Z}&\cd{LZ01}&\cd{z}&{\rm small z} \\ 
\cd{SP14}&\cd{;}&{\rm semicolon}&\cd{SM06}&\cd{{\Xlbracket}}&{\rm left square bracket}&\cd{SM11}&\cd{{\Xlbrace}}&{\rm left curly bracket} \\ 
\cd{SA03}&\cd{<}&{\rm less-than sign}&\cd{SM07}&\cd{{\Xbackslash}}&{\rm reverse solidus}&\cd{SM13}&\cd{|}&{\rm vertical bar} \\ 
\cd{SA04}&\cd{=}&{\rm equals sign}&\cd{SM08}&\cd{{\Xrbracket}}&{\rm right square bracket}&\cd{SM14}&\cd{{\Xrbrace}}&{\rm right curly bracket} \\ 
\cd{SA05}&\cd{>}&{\rm greater-than sign}&\cd{SD15}&\cd{{\Xcircumflex}}&{\rm circumflex accent}&\cd{SD19}&\cd{{\Xtilde}}&{\rm tilde} \\ 
\cd{SP15}&\cd{?}&{\rm question mark}&\cd{SP09}&\cd{{\Xunderscore}}&{\rm low line}&
\end{tabular*}
\vfill
\begin{small}
\noindent
The characters in this table plus the space and newline characters make up
the standard Common Lisp character repertoire (type \cd{standard-char}).
The character labels and character descriptions shown here are taken
from ISO standard 6937/2 .  The first character of the label
categorizes the character as Latin, Numeric, or Special.
\end{small}
\end{table}
}

\stepcounter{section}
\stepcounter{section}
\stepcounter{section}
\stepcounter{section}
\stepcounter{section}
\stepcounter{chapter}
{\newpage
\clearpage
\samepage \begin{tabular*}{\textwidth}{@{}l@{\extracolsep{\fill}}lll@{}}
elt&reverse&map&remove \\ 
length&nreverse&some&remove-duplicates \\ 
subseq&concatenate&every&delete \\ 
copy-seq&position&notany&delete-duplicates \\ 
fill&find&notevery&substitute \\ 
replace&sort&reduce&nsubstitute \\ 
count&merge&search&mismatch
\end{tabular*}
}

\stepcounter{section}
\stepcounter{section}
\stepcounter{section}
\stepcounter{section}
\stepcounter{section}
\stepcounter{chapter}
\stepcounter{section}
\stepcounter{section}
\stepcounter{section}
\stepcounter{section}
\stepcounter{section}
\stepcounter{section}
\stepcounter{chapter}
\stepcounter{section}
\stepcounter{section}
\stepcounter{chapter}
\stepcounter{section}
{\newpage
\clearpage
\samepage ${\it a}_2$
}

\stepcounter{section}
\stepcounter{section}
{\newpage
\clearpage
\samepage ${\it i}_0$
}

{\newpage
\clearpage
\samepage ${\it i}_1$
}

{\newpage
\clearpage
\samepage ${\it i}_{\hbox{\scriptsize\it n}-1}$
}

\stepcounter{section}
{\newpage
\clearpage
\samepage \begin{tabular*}{\textwidth}{@{}l@{\extracolsep{\fill}}lllll@{}}
~~~{\it argument1}~~&0&0&1&1 \\ 
~~~{\it argument2}~~&0&1&0&1&{\it Operation name} \\ 
\hlinesp
bit-and&0&0&0&1&{\rm and} \\ 
bit-ior&0&1&1&1&{\rm inclusive or} \\ 
bit-xor&0&1&1&0&{\rm exclusive or} \\ 
bit-eqv&1&0&0&1&{\rm equivalence (exclusive nor)} \\ 
bit-nand&1&1&1&0&{\rm not-and} \\ 
bit-nor&1&0&0&0&{\rm not-or} \\ 
bit-andc1&0&1&0&0&{\rm and complement of {\it argument1} with {\it argument2}} \\ 
bit-andc2&0&0&1&0&{\rm and {\it argument1} with complement of {\it argument2}} \\ 
bit-orc1&1&1&0&1&{\rm or complement of {\it argument1} with {\it argument2}} \\ 
bit-orc2&1&0&1&1&{\rm or {\it argument1} with complement of {\it argument2}} \\ 
\hline
\end{tabular*}
}

\stepcounter{section}
\stepcounter{section}
\stepcounter{chapter}
\stepcounter{section}
\stepcounter{section}
\stepcounter{section}
{\newpage
\clearpage
\samepage \begin{tabular*}{\textwidth}{@{}l@{\extracolsep{\fill}}ll@{}}
string= & string-equal & string-trim \\ 
string< & string-lessp &  string-left-trim \\ 
string> & string-greaterp &  string-right-trim \\ 
string<= & string-not-greaterp & string-upcase \\ 
string>= & string-not-lessp & string-downcase \\ 
string/= & string-not-equal & string-capitalize
\end{tabular*}
}

\stepcounter{chapter}
\stepcounter{section}
\stepcounter{section}
{\newpage
\clearpage
\samepage ${}_{\hbox{\scriptsize\it j}}$
}

\stepcounter{section}
\stepcounter{section}
\stepcounter{section}
\stepcounter{section}
\stepcounter{section}
\stepcounter{subsection}
\stepcounter{subsection}
\stepcounter{subsection}
\stepcounter{chapter}
\stepcounter{section}
\stepcounter{section}
{\newpage
\clearpage
\samepage \begin{tabular*}{\textwidth}{@{}l@{\extracolsep{\fill}}ll@{}}
+++ \EV\ (cons 'a 'b)&*** \EV\ NIL    &/// \EV\ () \\ 
++  \EV\ (hairy-loop)&**  \EV\ (A . B)&//  \EV\ ((A . B)) \\ 
+   \EV\ (floor 13 4)&*   \EV\ 3      &/   \EV\ (3 1)
\end{tabular*}
}

\stepcounter{chapter}
\stepcounter{section}
\stepcounter{section}
\stepcounter{section}
{\newpage
\clearpage
\samepage \begin{tabular*}{\textwidth}{@{\extracolsep{\fill}}lll@{}}
\cd{streamp} & \cd{pathname-host} & \cd{namestring} \\ 
\cd{pathname} & \cd{pathname-device} & \cd{file-namestring} \\ 
\cd{truename} & \cd{pathname-directory} & \cd{directory-namestring} \\ 
\cd{merge-pathnames} & \cd{pathname-name} & \cd{host-namestring} \\ 
\cd{open} & \cd{pathname-type} & \cd{enough-namestring} \\ 
\cd{probe-file} & \cd{pathname-version} & \cd{directory} \\ 
\end{tabular*}
}

\stepcounter{chapter}
\stepcounter{section}
\stepcounter{subsection}
{\newpage
\clearpage
\samepage \begin{table}% latex2html id marker 19926
\caption{Standard Character Syntax Types}
\label{Standard-Character-Syntax-Table}
\begin{tabular*}{\textwidth}{@{}l@{\extracolsep{\fill}}ll@{}}
$\langle$tab$\rangle$\cd{~~}{\it whitespace}&$\langle$page$\rangle$\cd{~~}{\it whitespace}&$\langle$newline$\rangle$\cd{~~}{\it whitespace} \\ 
$\langle$space$\rangle$\cd{~~}{\it whitespace}&\cd{{\Xatsign}~~}{\it constituent}&\cd{{\Xbq}~~}{\it terminating macro} \\ 
\cd{!~~}{\it constituent} *&\cd{A~~}{\it constituent}&\cd{a~~}{\it constituent} \\ 
\cd{"~~}{\it terminating macro}&\cd{B~~}{\it constituent}&\cd{b~~}{\it constituent} \\ 
\cd{\#~~}{\it non-terminating macro}&\cd{C~~}{\it constituent}&\cd{c~~}{\it constituent} \\ 
\cd{\$~~}{\it constituent}&\cd{D~~}{\it constituent}&\cd{d~~}{\it constituent} \\ 
\cd{\%~~}{\it constituent}&\cd{E~~}{\it constituent}&\cd{e~~}{\it constituent} \\ 
\cd{\&~~}{\it constituent}&\cd{F~~}{\it constituent}&\cd{f~~}{\it constituent} \\ 
\cd{'~~}{\it terminating macro}&\cd{G~~}{\it constituent}&\cd{g~~}{\it constituent} \\ 
\cd{(~~}{\it terminating macro}&\cd{H~~}{\it constituent}&\cd{h~~}{\it constituent} \\ 
\cd{)~~}{\it terminating macro}&\cd{I~~}{\it constituent}&\cd{i~~}{\it constituent} \\ 
\cd{*~~}{\it constituent}&\cd{J~~}{\it constituent}&\cd{j~~}{\it constituent} \\ 
\cd{+~~}{\it constituent}&\cd{K~~}{\it constituent}&\cd{k~~}{\it constituent} \\ 
\cd{,~~}{\it terminating macro}&\cd{L~~}{\it constituent}&\cd{l~~}{\it constituent} \\ 
\cd{-~~}{\it constituent}&\cd{M~~}{\it constituent}&\cd{m~~}{\it constituent} \\ 
\cd{.~~}{\it constituent}&\cd{N~~}{\it constituent}&\cd{n~~}{\it constituent} \\ 
\cd{/~~}{\it constituent}&\cd{O~~}{\it constituent}&\cd{o~~}{\it constituent} \\ 
\cd{0~~}{\it constituent}&\cd{P~~}{\it constituent}&\cd{p~~}{\it constituent} \\ 
\cd{1~~}{\it constituent}&\cd{Q~~}{\it constituent}&\cd{q~~}{\it constituent} \\ 
\cd{2~~}{\it constituent}&\cd{R~~}{\it constituent}&\cd{r~~}{\it constituent} \\ 
\cd{3~~}{\it constituent}&\cd{S~~}{\it constituent}&\cd{s~~}{\it constituent} \\ 
\cd{4~~}{\it constituent}&\cd{T~~}{\it constituent}&\cd{t~~}{\it constituent} \\ 
\cd{5~~}{\it constituent}&\cd{U~~}{\it constituent}&\cd{u~~}{\it constituent} \\ 
\cd{6~~}{\it constituent}&\cd{V~~}{\it constituent}&\cd{v~~}{\it constituent} \\ 
\cd{7~~}{\it constituent}&\cd{W~~}{\it constituent}&\cd{w~~}{\it constituent} \\ 
\cd{8~~}{\it constituent}&\cd{X~~}{\it constituent}&\cd{x~~}{\it constituent} \\ 
\cd{9~~}{\it constituent}&\cd{Y~~}{\it constituent}&\cd{y~~}{\it constituent} \\ 
\cd{:~~}{\it constituent}&\cd{Z~~}{\it constituent}&\cd{z~~}{\it constituent} \\ 
\cd{;~~}{\it terminating macro}&\cd{{\Xlbracket}~~}{\it constituent} *&\cd{{\Xlbrace}~~}{\it constituent} * \\ 
\cd{<~~}{\it constituent}&\cd{{\Xbackslash}~~}{\it single escape}&\cd{|~~}{\it multiple escape} \\ 
\cd{=~~}{\it constituent}&\cd{{\Xrbracket}~~}{\it constituent} *&\cd{{\Xrbrace}~~}{\it constituent} * \\ 
\cd{>~~}{\it constituent}&\cd{{\Xcircumflex}~~}{\it constituent}&\cd{{\Xtilde}~~}{\it constituent} \\ 
\cd{?~~}{\it constituent} *&\cd{{\Xunderscore}~~}{\it constituent}&$\langle$rubout$\rangle$\cd{~~}{\it constituent} \\ 
$\langle$backspace$\rangle$\cd{~~}{\it constituent}&$\langle$return$\rangle$\cd{~~}{\it whitespace}&$\langle$linefeed$\rangle$\cd{~~}{\it whitespace}
\end{tabular*}
\vfill
\begin{small}
\noindent
The characters marked with an asterisk are initially constituents
but are reserved to the user for use as macro characters or for
any other desired purpose.
\end{small}
\end{table}
}

\stepcounter{subsection}
{\newpage
\clearpage
\samepage \begin{table}% latex2html id marker 20349
[t]
\caption{Actual Syntax of Numbers}
\label{NUMBER-SYNTAX-TABLE}
\tabbingsep=0pt
\normalsize
\begin{tabbing}
{\it number} ::= {\it integer} {\Mor} {\it ratio} {\Mor} {\it floating-point-number} \\ 
{\it integer} ::= \Mopt{{\it sign}} \Mplus{{\it digit}} \Mopt{{\it decimal-point}} \\ 
{\it ratio} ::= \Mopt{{\it sign}} \Mplus{{\it digit}} \cd{/} \Mplus{{\it digit}} \\ 
{\it floating-point-number} ::=\= \Mopt{{\it sign}} \Mstar{{\it digit}} {\it decimal-point} \Mplus{{\it digit}} \Mopt{{\it exponent}} \\ 
\>{\Mor} \'\Mopt{{\it sign}} \Mplus{{\it digit}} \Mopt{{\it decimal-point}
\Mstar{{\it digit}}} {\it exponent} \\  {\it sign} ::= \cd{+} {\Mor} \cd{-} \\ 
{\it decimal-point} ::= \cd{.} \\ 
{\it digit} ::= \cd{0} {\Mor} \cd{1} {\Mor} \cd{2} {\Mor} \cd{3} {\Mor} \cd{4}
         {\Mor} \cd{5} {\Mor} \cd{6} {\Mor} \cd{7} {\Mor} \cd{8} {\Mor} \cd{9} \\ 
{\it exponent} ::= {\it exponent-marker} \Mopt{{\it sign}} \Mplus{{\it digit}} \\ 
{\it exponent-marker} ::= \cd{e} {\Mor} \cd{s} {\Mor} \cd{f} {\Mor} \cd{d} {\Mor} \cd{l}
                   {\Mor} \cd{E} {\Mor} \cd{S} {\Mor} \cd{F} {\Mor} \cd{D} {\Mor} \cd{L}
\end{tabbing}
\end{table}
}

{\newpage
\clearpage
\samepage \begin{table}% latex2html id marker 20442
\caption{Standard Constituent Character Attributes}
\label{Standard-Readtable-Attributes-Table}
\begin{tabular*}{\textwidth}{@{\extracolsep{\fill}}l@{\extracolsep{\fill}}lllll@{}}
\cd{!}&{\it alphabetic}&$\langle$page$\rangle$&{\it illegal}&$\langle$backspace$\rangle$&{\it illegal} \\ 
\cd{"}&{\it alphabetic} *&$\langle$return$\rangle$&{\it illegal} *&$\langle$tab$\rangle$&{\it illegal} * \\ 
\cd{\#}&{\it alphabetic} *&$\langle$space$\rangle$&{\it illegal} *&$\langle$newline$\rangle$&{\it illegal} * \\ 
\cd{\$}&{\it alphabetic}&$\langle$rubout$\rangle$&{\it illegal}&$\langle$linefeed$\rangle$&{\it illegal} * \\ 
\cd{\%}&{\it alphabetic}&\cd{.}&\multicolumn{3}{l}{{\it alphabetic}, {\it dot}, {\it decimal point}}\\ 
\cd{\&}&{\it alphabetic}&\cd{+}&\multicolumn{3}{l}{{\it alphabetic}, {\it plus sign}} \\ 
\cd{'}&{\it alphabetic} *&\cd{-}&\multicolumn{3}{l}{{\it alphabetic}, {\it minus sign}} \\ 
\cd{(}&{\it alphabetic} *&\cd{*}&{\it alphabetic} \\ 
\cd{)}&{\it alphabetic} *&\cd{/}&\multicolumn{3}{l}{{\it alphabetic}, {\it ratio marker}} \\ 
\cd{,}&{\it alphabetic} *&\cd{{\Xatsign}}&{\it alphabetic} \\ 
\cd{0}&{\it alphadigit}&\cd{A}, \cd{a}&{\it alphadigit} \\ 
\cd{1}&{\it alphadigit}&\cd{B}, \cd{b}&{\it alphadigit} \\ 
\cd{2}&{\it alphadigit}&\cd{C}, \cd{c}&{\it alphadigit} \\ 
\cd{3}&{\it alphadigit}&\cd{D}, \cd{d}&\multicolumn{3}{l}{{\it alphadigit}, {\it double-float exponent marker}} \\ 
\cd{4}&{\it alphadigit}&\cd{E}, \cd{e}&\multicolumn{3}{l}{{\it alphadigit}, {\it float exponent marker}} \\ 
\cd{5}&{\it alphadigit}&\cd{F}, \cd{f}&\multicolumn{3}{l}{{\it alphadigit}, {\it single-float exponent marker}} \\ 
\cd{6}&{\it alphadigit}&\cd{G}, \cd{g}&{\it alphadigit} \\ 
\cd{7}&{\it alphadigit}&\cd{H}, \cd{h}&{\it alphadigit} \\ 
\cd{8}&{\it alphadigit}&\cd{I}, \cd{i}&{\it alphadigit} \\ 
\cd{9}&{\it alphadigit}&\cd{J}, \cd{j}&{\it alphadigit} \\ 
\cd{:}&{\it package marker}~~~~~~&\cd{K}, \cd{k}&{\it alphadigit} \\ 
\cd{;}&{\it alphabetic} *&\cd{L}, \cd{l}&\multicolumn{3}{l}{{\it alphadigit}, {\it long-float exponent marker}} \\ 
\cd{<}&{\it alphabetic}&\cd{M}, \cd{m}&{\it alphadigit} \\ 
\cd{=}&{\it alphabetic}&\cd{N}, \cd{n}&{\it alphadigit} \\ 
\cd{>}&{\it alphabetic}&\cd{O}, \cd{o}&{\it alphadigit} \\ 
\cd{?}&{\it alphabetic}&\cd{P}, \cd{p}&{\it alphadigit} \\ 
\cd{{\Xlbracket}}&{\it alphabetic}&\cd{Q}, \cd{q}&{\it alphadigit} \\ 
\cd{{\Xbackslash}}&{\it alphabetic} *&\cd{R}, \cd{r}&{\it alphadigit} \\ 
\cd{{\Xrbracket}}&{\it alphabetic}&\cd{S}, \cd{s}&\multicolumn{3}{l}{{\it alphadigit}, {\it short-float exponent marker}} \\ 
\cd{{\Xcircumflex}}&{\it alphabetic}&\cd{T}, \cd{t}&{\it alphadigit} \\ 
\cd{{\Xunderscore}}&{\it alphabetic}&\cd{U}, \cd{u}&{\it alphadigit} \\ 
\cd{{\Xbq}}&{\it alphabetic} *&\cd{V}, \cd{v}&{\it alphadigit} \\ 
\cd{{\Xlbrace}}&{\it alphabetic}&\cd{W}, \cd{w}&{\it alphadigit} \\ 
\cd{|}&{\it alphabetic} *&\cd{X}, \cd{x}&{\it alphadigit} \\ 
\cd{{\Xrbrace}}&{\it alphabetic}&\cd{Y}, \cd{y}&{\it alphadigit} \\ 
\cd{{\Xtilde}}&{\it alphabetic}&\cd{Z}, \cd{z}&{\it alphadigit} \\ 
\end{tabular*}
\vfill
\begin{footnotesize}
\noindent
These interpretations apply only to characters whose
syntactic type is {\it constituent}.  Entries marked
with an asterisk are normally shadowed because the characters
are of syntactic type
{\it whitespace}, {\it macro}, {\it single escape}, or {\it multiple escape}.
An {\it alphadigit} character is interpreted as a
digit if it is a valid digit in the radix specified by {\small \cd{*read-base*}};
otherwise it is alphabetic.
Characters with an {\it illegal} attribute can never appear in
a token except under the control of an escape character.
\end{footnotesize}
\end{table}
}

\stepcounter{subsection}
\stepcounter{subsection}
{\newpage
\clearpage
\samepage \begin{table}% latex2html id marker 21146
\caption{Standard \# Macro Character Syntax}
\label{Standard-Sharp-Macro-Definitions-Table}
\begin{tabular*}{\textwidth}{@{\extracolsep{\fill}}l@{\extracolsep{\fill}}lll@{}}
\cd{\#!}&{\it undefined} *&\cd{\#}$\langle$backspace$\rangle$&{\it signals error} \\ 
\cd{\#"}&{\it undefined}&\cd{\#}$\langle$tab$\rangle$&{\it signals error} \\ 
\cd{\#\#}&{\it reference to \cd{\#=} label}&\cd{\#}$\langle$newline$\rangle$&{\it signals error} \\ 
\cd{\#\$}&{\it undefined}&\cd{\#}$\langle$linefeed$\rangle$&{\it signals error} \\ 
\cd{\#\%}&{\it undefined}&\cd{\#}$\langle$page$\rangle$&{\it signals error} \\ 
\cd{\#\&}&{\it undefined}&\cd{\#}$\langle$return$\rangle$&{\it signals error} \\ 
\cd{\#'}&{\it \cd{function} abbreviation}&\cd{\#}$\langle$space$\rangle$&{\it signals error} \\ 
\cd{\#(}&{\it simple vector}&\cd{\#+}&{\it read-time conditional} \\ 
\cd{\#)}&{\it signals error}&\cd{\#-}&{\it read-time conditional} \\ 
\cd{\#*}&{\it bit-vector}&\cd{\#.}&{\it read-time evaluation} \\ 
\cd{\#,}&{\it load-time evaluation}&\cd{\#/}&{\it undefined} \\ 
\cd{\#0}&{\it used for infix arguments}~~~~~~&\cd{\#A}, \cd{\#a}&{\it array} \\ 
\cd{\#1}&{\it used for infix arguments}&\cd{\#B}, \cd{\#b}&{\it binary rational} \\ 
\cd{\#2}&{\it used for infix arguments}&\cd{\#C}, \cd{\#c}&{\it complex number} \\ 
\cd{\#3}&{\it used for infix arguments}&\cd{\#D}, \cd{\#d}&{\it undefined} \\ 
\cd{\#4}&{\it used for infix arguments}&\cd{\#E}, \cd{\#e}&{\it undefined} \\ 
\cd{\#5}&{\it used for infix arguments}&\cd{\#F}, \cd{\#f}&{\it undefined} \\ 
\cd{\#6}&{\it used for infix arguments}&\cd{\#G}, \cd{\#g}&{\it undefined} \\ 
\cd{\#7}&{\it used for infix arguments}&\cd{\#H}, \cd{\#h}&{\it undefined} \\ 
\cd{\#8}&{\it used for infix arguments}&\cd{\#I}, \cd{\#i}&{\it undefined} \\ 
\cd{\#9}&{\it used for infix arguments}&\cd{\#J}, \cd{\#j}&{\it undefined} \\ 
\cd{\#:}&{\it uninterned symbol}&\cd{\#K}, \cd{\#k}&{\it undefined} \\ 
\cd{\#;}&{\it undefined}&\cd{\#L}, \cd{\#l}&{\it undefined} \\ 
\cd{\#<}&{\it signals error}&\cd{\#M}, \cd{\#m}&{\it undefined} \\ 
\cd{\#=}&{\it label following object}&\cd{\#N}, \cd{\#n}&{\it undefined} \\ 
\cd{\#>}&{\it undefined}&\cd{\#O}, \cd{\#o}&{\it octal rational} \\ 
\cd{\#?}&{\it undefined} *&\cd{\#P}, \cd{\#p}&{\it pathname} \\ 
\cd{\#{\Xatsign}}&{\it undefined}&\cd{\#Q}, \cd{\#q}&{\it undefined} \\ 
\cd{\#{\Xlbracket}}&{\it undefined} *&\cd{\#R}, \cd{\#r}&{\it radix-n rational} \\ 
\cd{\#{\Xbackslash}}&{\it character object}&\cd{\#S}, \cd{\#s}&{\it structure} \\ 
\cd{\#{\Xrbracket}}&{\it undefined} *&\cd{\#T}, \cd{\#t}&{\it undefined} \\ 
\cd{\#{\Xcircumflex}}&{\it undefined}&\cd{\#U}, \cd{\#u}&{\it undefined} \\ 
\cd{\#{\Xunderscore}}&{\it undefined}&\cd{\#V}, \cd{\#v}&{\it undefined} \\ 
\cd{\#{\Xbq}}&{\it undefined}&\cd{\#W}, \cd{\#w}&{\it undefined} \\ 
\cd{\#{\Xlbrace}}&{\it undefined} *&\cd{\#X}, \cd{\#x}&{\it hexadecimal rational} \\ 
\cd{\#|}&{\it balanced comment}&\cd{\#Y}, \cd{\#y}&{\it undefined} \\ 
\cd{\#{\Xrbrace}}&{\it undefined} *&\cd{\#Z}, \cd{\#z}&{\it undefined} \\ 
\cd{\#{\Xtilde}}&{\it undefined}&\cd{\#}$\langle$rubout$\rangle$&{\it undefined}
\end{tabular*}
\vfill
\begin{small}
\noindent
The combinations marked by an asterisk are explicitly reserved to the user
and will never be defined by Common Lisp.

\begin{new}
X3J13 voted in June 1989 \issue{PATHNAME-PRINT-READ} to
specify \cd{\#P} and \cd{\#p} ({\it undefined}
in the first edition).
\end{new}
\end{small}
\end{table}
}

\stepcounter{subsection}
{\newpage
\clearpage
\samepage \Xarrowright
}

{\newpage
\clearpage
\samepage \begin{table}% latex2html id marker 21931
\begin{new}
\caption{Macro Character Definition for Xapping Syntax}
\label{XAPPING-MACRO-CHAR-TABLE}
\begin{lisp}
(defun open-brace-macro-char (s macro-char) \\ 
~~(declare (ignore macro-char)) \\ 
~~(do ((ch (peek-char t s t nil t) (peek-char t s t nil t)) \\ 
~~~~~~~(domain '())~~(range '())~~(exceptions '())) \\ 
~~~~~~((char= ch \#{\Xbackslash}{\Xrbrace}) \\ 
~~~~~~~(read-char s t nil t) \\ 
~~~~~~~(construct-xapping (reverse domain) (reverse range))) \\ 
~~~~(cond ((char= ch \#{\Xbackslash}{\Xarrowright}) \\ 
~~~~~~~~~~~(read-char s t nil t) \\ 
~~~~~~~~~~~(let ((nextch (peek-char nil s t nil t))) \\ 
~~~~~~~~~~~~~(cond ((char= nextch \#{\Xbackslash}{\Xrbrace}) \\ 
~~~~~~~~~~~~~~~~~~~~(read-char s t nil t) \\ 
~~~~~~~~~~~~~~~~~~~~(return (xap (reverse domain) \\ 
~~~~~~~~~~~~~~~~~~~~~~~~~~~~~~~~~(reverse range) \\ 
~~~~~~~~~~~~~~~~~~~~~~~~~~~~~~~~~nil :universal exceptions))) \\ 
~~~~~~~~~~~~~~~~~~~(t (let ((item (read s t nil t))) \\ 
~~~~~~~~~~~~~~~~~~~~~~~~(cond ((char= (peek-char t s t nil t) \#{\Xbackslash}{\Xrbrace}) \\ 
~~~~~~~~~~~~~~~~~~~~~~~~~~~~~~~(read-char s t nil t) \\ 
~~~~~~~~~~~~~~~~~~~~~~~~~~~~~~~(return (xap (reverse domain) \\ 
~~~~~~~~~~~~~~~~~~~~~~~~~~~~~~~~~~~~~~~~~~~~(reverse range) \\ 
~~~~~~~~~~~~~~~~~~~~~~~~~~~~~~~~~~~~~~~~~~~~item :constant \\ 
~~~~~~~~~~~~~~~~~~~~~~~~~~~~~~~~~~~~~~~~~~~~exceptions))) \\ 
~~~~~~~~~~~~~~~~~~~~~~~~~~~~~~(t (reader-error s \\ 
~~~~~~~~~~~~~~~~~~~~~~~~~~~~~~~~~~~"Default~{\Xarrowright} item must be last")))))))) \\ 
~~~~~~~~~~(t (let ((item (read-preserving-whitespace s t nil t)) \\ 
~~~~~~~~~~~~~~~~~~~(nextch (peek-char nil s t nil t))) \\ 
~~~~~~~~~~~~~~~(cond ((char= nextch \#{\Xbackslash}{\Xarrowright}) \\ 
~~~~~~~~~~~~~~~~~~~~~~(read-char s t nil t) \\ 
~~~~~~~~~~~~~~~~~~~~~~(cond ((member (peek-char nil s t nil t) \\ 
~~~~~~~~~~~~~~~~~~~~~~~~~~~~~~~~~~~~~'(\#{\Xbackslash}Space \#{\Xbackslash}Tab \#{\Xbackslash}Newline)) \\ 
~~~~~~~~~~~~~~~~~~~~~~~~~~~~~(push item exceptions)) \\ 
~~~~~~~~~~~~~~~~~~~~~~~~~~~~(t (push item domain) \\ 
~~~~~~~~~~~~~~~~~~~~~~~~~~~~~~~(push (read s t nil t) range)))) \\ 
~~~~~~~~~~~~~~~~~~~~~((char= nch \#{\Xbackslash}{\Xrbrace}) \\ 
~~~~~~~~~~~~~~~~~~~~~~(read-char s t nil t) \\ 
~~~~~~~~~~~~~~~~~~~~~~(push item domain) \\ 
~~~~~~~~~~~~~~~~~~~~~~(push item range) \\ 
~~~~~~~~~~~~~~~~~~~~~~(return (xap (reverse domain) (reverse range)))) \\ 
~~~~~~~~~~~~~~~~~~~~~(t (push item domain) \\ 
~~~~~~~~~~~~~~~~~~~~~~~~(push item range))))))))
\end{lisp}
\end{new}
\end{table}
}

\stepcounter{subsection}
{\newpage
\clearpage
\samepage $10^{-3}$
}

{\newpage
\clearpage
\samepage $10^7$
}

{\newpage
\clearpage
\samepage \begin{table}% latex2html id marker 22672
[t]
\caption{Examples of Print Level and Print Length Abbreviation}
\label{LEVEL-LENGTH-TABLE}
\begin{lisp}
\begin{tabular*}{\textwidth}{@{}l@{\extracolsep{\fill}}@{}ll@{}}
{\it v}&{\it n}&Output \\ 
\hlinesp
0&1&\# \\ 
1&1&(if ...) \\ 
1&2&(if \# ...) \\ 
1&3&(if \# \# ...) \\ 
1&4&(if \# \# \#) \\ 
2&1&(if ...) \\ 
2&2&(if (member x ...) ...) \\ 
2&3&(if (member x y) (+ \# 3) ...) \\ 
3&2&(if (member x ...) ...) \\ 
3&3&(if (member x y) (+ (car x) 3) ...) \\ 
3&4&(if (member x y) (+ (car x) 3) '(foo . \#(a b c d ...))) \\ 
3&5&(if (member x y) (+ (car x) 3) '(foo . \#(a b c d "Baz")))
\end{tabular*}
\end{lisp}
\end{table}
}

{\newpage
\clearpage
\samepage \begin{table}% latex2html id marker 22738
[t]
\caption{Standard Bindings for I/O Control Variables}
\label{WITH-STANDARD-IO-SYNTAX-TABLE}
\begin{flushleft}
\cf
\begin{tabular}{@{}ll@{}}
{\rm Variable}&{\rm Value} \\ 
\hlinesp
      {*package*}                      &     {\rm the \cd{common-lisp-user} package} \\ 
      {*print-array*}                  &     t \\ 
      {*print-base*}                   &     10 \\ 
      {*print-case*}                   &     :upcase \\ 
      {*print-circle*}                 &     nil \\ 
      {*print-escape*}                 &     t \\ 
      {*print-gensym*}                 &     t \\ 
      {*print-length*}                 &     nil \\ 
      {*print-level*}                  &     nil \\ 
      {*print-lines*}                  &     nil {\rm *} \\ 
      {*print-miser-width*}            &     nil {\rm *} \\ 
      {*print-pprint-dispatch*}        &     nil {\rm *} \\ 
      {*print-pretty*}                 &     nil \\ 
      {*print-radix*}                  &     nil \\ 
      {*print-readably*}               &     t \\ 
      {*print-right-margin*}           &     nil {\rm *} \\ 
      {*read-base*}                    &     10 \\ 
      {*read-default-float-format*}    &     single-float \\ 
      {*read-eval*}                    &     t \\ 
      {*read-suppress*}                &     nil \\ 
      {*readtable*}                    &     {\rm the standard readtable}
\end{tabular}
\end{flushleft}
* X3J13 voted in June 1989 \issue{PRETTY-PRINT-INTERFACE}
to introduce the printer control variables
\cd{*print-\discretionary{}{}{}right-\discretionary{}{}{}margin*},
\cd{*print-\discretionary{}{}{}miser-\discretionary{}{}{}width*},
\cd{*print-\discretionary{}{}{}lines*},
and \cd{*print-\discretionary{}{}{}pprint-\discretionary{}{}{}dispatch*}
(see section~\ref{PPRINT-VARIABLES-SECTION})
but did not specify the values to which \cd{with-standard-io-syntax}
should bind them.  I recommend that all four should be bound to \cd{nil}.
\end{table}
}

\stepcounter{section}
\stepcounter{subsection}
\stepcounter{subsection}
\stepcounter{section}
\stepcounter{subsection}
\stepcounter{subsection}
\stepcounter{subsection}
{\newpage
\clearpage
\samepage $10^60$
}

{\newpage
\clearpage
\samepage $2^{200}$
}

{\newpage
\clearpage
\samepage $10^{3(\hbox{\scriptsize\it k}+1)}$
}

{\newpage
\clearpage
\samepage $10^{12}$
}

{\newpage
\clearpage
\samepage $10^{6\hbox{\scriptsize\it k}}$
}

{\newpage
\clearpage
\samepage $10^{18}$
}

{\newpage
\clearpage
\samepage $10^{\hbox{\scriptsize\it k}}$
}

{\newpage
\clearpage
\samepage \(10^{\hbox{\scriptsize\it n}-1}\leq\hbox{\it arg}<10^{\hbox{\scriptsize\it n}}\)
}

{\newpage
\clearpage
\samepage \Xarrowdown
}

{\newpage
\clearpage
\samepage \begin{table}% latex2html id marker 25237
[t]
\caption{Print Function for the Xapping Data Type}
\label{XAPPING-FORMAT-TABLE}
\begingroup
\small
\begin{lisp}
(defun print-xapping (xapping stream depth) \\ 
~~(declare (ignore depth)) \\ 
~~(format stream \\ 
~~~~~~~~~~;; Are you ready for this one? \\ 
~~~~~~~~~~"{\Xtilde}:[{\Xlbrace}{\Xtilde};[{\Xtilde}]{\Xtilde}:{\Xlbrace}{\Xtilde}S{\Xtilde}:[{\Xarrowright}{\Xtilde}S{\Xtilde};{\Xtilde}*{\Xtilde}]{\Xtilde}:{\Xcircumflex} {\Xtilde}{\Xrbrace}{\Xtilde}:[{\Xtilde}; {\Xtilde}]{\Xtilde} \\ 
~~~~~~~~~~~{\Xtilde}{\Xlbrace}{\Xtilde}S{\Xarrowright}{\Xtilde}{\Xcircumflex} {\Xtilde}{\Xrbrace}{\Xtilde}:[{\Xtilde}; {\Xtilde}]{\Xtilde}[{\Xtilde}*{\Xtilde};{\Xarrowright}{\Xtilde}S{\Xtilde};{\Xarrowright}{\Xtilde}*{\Xtilde}]{\Xtilde}:[{\Xrbrace}{\Xtilde};]{\Xtilde}]" \\ 
~~~~~~~~~~;; Is that clear? \\ 
~~~~~~~~~~(xectorp xapping) \\ 
~~~~~~~~~~(do ((vp (xectorp xapping)) \\ 
~~~~~~~~~~~~~~~(sp (finite-part-is-xetp xapping)) \\ 
~~~~~~~~~~~~~~~(d (xapping-domain xapping) (cdr d)) \\ 
~~~~~~~~~~~~~~~(r (xapping-range xapping) (cdr r)) \\ 
~~~~~~~~~~~~~~~(z '() (cons (list (if vp (car r) (car d)) \\ 
~~~~~~~~~~~~~~~~~~~~~~~~~~~~~~~~~~(or vp sp) \\ 
~~~~~~~~~~~~~~~~~~~~~~~~~~~~~~~~~~(car r)) \\ 
~~~~~~~~~~~~~~~~~~~~~~~~~~~~z))) \\ 
~~~~~~~~~~~~~~((null d) (reverse z))) \\ 
~~~~~~~~~~(and (xapping-domain xapping) \\ 
~~~~~~~~~~~~~~~(or (xapping-exceptions xapping) \\ 
~~~~~~~~~~~~~~~~~~~(xapping-infinite xapping))) \\ 
~~~~~~~~~~(xapping-exceptions xapping) \\ 
~~~~~~~~~~(and (xapping-exceptions xapping) \\ 
~~~~~~~~~~~~~~~(xapping-infinite xapping)) \\ 
~~~~~~~~~~(ecase (xapping-infinite xapping) \\ 
~~~~~~~~~~~~((nil) 0) \\ 
~~~~~~~~~~~~(:constant 1) \\ 
~~~~~~~~~~~~(:universal 2)) \\ 
~~~~~~~~~~(xapping-default xapping) \\ 
~~~~~~~~~~(xectorp xapping)))
\end{lisp}
\endgroup
See section~\ref{READTABLE-SECTION} for the \cd{defstruct} definition of the \cd{xapping} data
type, whose accessor functions are used in this code.
\end{table}
}

{\newpage
\clearpage
\samepage \begin{tabular*}{\textwidth}{@{}l@{\extracolsep{\fill}}p{17pc}@{}}
\cd{{\Xtilde}:[{\Xlbrace}{\Xtilde};[{\Xtilde}]}&
Print ``\cd{[}'' for a xector, and ``\cd{{\Xlbrace}}'' otherwise. \\ 
\cd{{\Xtilde}:{\Xlbrace}{\Xtilde}S{\Xtilde}:[{\Xarrowright}{\Xtilde}S{\Xtilde};{\Xtilde}*{\Xtilde}]{\Xtilde}:{\Xcircumflex} {\Xtilde}{\Xrbrace}}& 
Given a list of lists, print the pairs.  Each sublist has three elements:
the index (or the value if we're printing a xector); a flag that is true for
either a xector or xet (in which case no arrow is printed);
and the value.  Note the use of \cd{{\Xtilde}:{\Xlbrace}} to iterate, and the use
of \cd{{\Xtilde}:{\Xcircumflex}} to avoid printing a separating space after the
final pair (or at all, if there are no pairs). \\ 
\cd{{\Xtilde}:[{\Xtilde}; {\Xtilde}]}&
If there were pairs and there are exceptions or an infinite part, print a separating space. \\ 
\cd{{\Xtilde}$\langle${\rm newline}$\rangle$}&
Do nothing.  This merely allows the format control string to be broken across two lines. \\ 
\cd{{\Xtilde}{\Xlbrace}{\Xtilde}S{\Xarrowright}{\Xtilde}{\Xcircumflex} {\Xtilde}{\Xrbrace}}&
Given a list of exception indices, print them.
Note the use of \cd{{\Xtilde}{\Xlbrace}} to iterate, and the use
of \cd{{\Xtilde}{\Xcircumflex}} to avoid printing a separating space after the
final exception (or at all, if there are no exceptions). \\ 
\cd{{\Xtilde}:[{\Xtilde}; {\Xtilde}]}&
If there were exceptions and there is an infinite part, print a separating space. \\ 
\cd{{\Xtilde}[{\Xtilde}*{\Xtilde};{\Xarrowright}{\Xtilde}S{\Xtilde};{\Xarrowright}{\Xtilde}*{\Xtilde}]}&
Use \cd{{\Xtilde}[} to choose one of three cases for printing the infinite part. \\ 
\cd{{\Xtilde}:[{\Xrbrace}{\Xtilde};]{\Xtilde}]}&
Print ``\cd{]}'' for a xector, and ``\cd{{\Xrbrace}}'' otherwise.
\end{tabular*}
}

\stepcounter{section}
\stepcounter{chapter}
\stepcounter{section}
\stepcounter{subsection}
\stepcounter{subsection}
\stepcounter{subsection}
\stepcounter{subsection}
\stepcounter{subsection}
\stepcounter{subsubsection}
\stepcounter{subsubsection}
{\newpage
\clearpage
\samepage \begin{tabular*}{\linewidth}{@{\extracolsep{\fill}}lll@{}}
  \cd{compile-file} & \cd{file-write-date} & \cd{pathname-name} \\ 
  \cd{compile-file-pathname} & \cd{host-namestring} & \cd{pathname-type} \\ 
  \cd{delete-file} & \cd{load} & \cd{pathname-version} \\ 
  \cd{directory} & \cd{namestring} & \cd{probe-file} \\ 
  \cd{directory-namestring} & \cd{open} & \cd{rename-file} \\ 
  \cd{dribble} & \cd{pathname} & \cd{translate-pathname} \\ 
  \cd{ed} & \cd{pathname-device} & \cd{truename} \\ 
  \cd{enough-namestring} & \cd{pathname-directory} & \cd{wild-pathname-p} \\ 
  \cd{file-author} & \cd{pathname-host} & \cd{with-open-file} \\ 
  \cd{file-namestring} & \cd{pathname-match-p} &
  \end{tabular*}
}

\stepcounter{subsubsection}
{\newpage
\clearpage
\samepage \begin{tabular*}{\linewidth}{@{}l@{\extracolsep{\fill}}ll@{}}
  \cd{compile-file} & \cd{ed} & \cd{probe-file} \\ 
  \cd{compile-file-pathname} & \cd{file-author} & \cd{rename-file} \\ 
  \cd{delete-file} & \cd{file-write-date} & \cd{truename} \\ 
  \cd{directory} & \cd{load} & \cd{with-open-file} \\ 
  \cd{dribble} & \cd{open} & 
  \end{tabular*}
}

\stepcounter{subsubsection}
\stepcounter{subsubsection}
\stepcounter{subsection}
{\newpage
\clearpage
\samepage \begin{tabular*}{\textwidth}{@{\extracolsep{\fill}}lll@{}}
\cd{pathname} & \cd{pathname-device} & \cd{namestring} \\ 
\cd{truename} & \cd{pathname-directory} & \cd{file-namestring} \\ 
\cd{parse-namestring} & \cd{pathname-name} & \cd{directory-namestring} \\ 
\cd{merge-pathnames} & \cd{pathname-type} & \cd{host-namestring} \\ 
\cd{pathname-host} & \cd{pathname-version} & \cd{enough-namestring}
\end{tabular*}
}

{\newpage
\clearpage
\samepage \begin{tabular*}{\textwidth}{@{\extracolsep{\fill}}lll@{}}
\cd{open} & \cd{rename-file~~~~~~~} & \cd{file-write-date~~~~~} \\ 
\cd{with-open-file~} & \cd{delete-file} & \cd{file-author} \\ 
\cd{load} & \cd{probe-file} & \cd{directory} \\ 
\cd{compile-file}
\end{tabular*}
}

{\newpage
\clearpage
\samepage \begin{tabular*}{\textwidth}{@{\extracolsep{\fill}}ll@{}}
\cd{make-two-way-stream} & \cd{make-string-input-stream} \\ 
\cd{make-echo-stream} & \cd{make-string-output-stream} \\ 
\cd{make-broadcast-stream} & \cd{with-input-from-string} \\ 
\cd{make-concatenated-stream} & \cd{with-output-to-string}
\end{tabular*}
}

\stepcounter{section}
\stepcounter{section}
\stepcounter{section}
\stepcounter{section}
\stepcounter{chapter}
\stepcounter{section}
\stepcounter{section}
\stepcounter{section}
\stepcounter{chapter}
\stepcounter{section}
{\newpage
\clearpage
\samepage ${\it b}^2-4{\it a}{\it c}$
}

\stepcounter{subsection}
\stepcounter{subsection}
\stepcounter{subsection}
\stepcounter{subsection}
{\newpage
\clearpage
\samepage ${\it S}'$
}

{\newpage
\clearpage
\samepage ${\it A}'$
}

{\newpage
\clearpage
\samepage ${\it B}'$
}

\stepcounter{section}
\stepcounter{section}
\stepcounter{section}
\stepcounter{subsection}
\stepcounter{subsection}
\stepcounter{section}
\stepcounter{chapter}
\stepcounter{section}
\stepcounter{section}
\stepcounter{section}
\stepcounter{subsection}
\stepcounter{subsection}
\stepcounter{subsection}
\stepcounter{section}
\stepcounter{section}
\stepcounter{section}
\stepcounter{section}
\stepcounter{section}
\stepcounter{section}
\stepcounter{section}
\stepcounter{section}
\stepcounter{section}
\stepcounter{subsection}
\stepcounter{subsection}
\stepcounter{chapter}
\stepcounter{section}
\stepcounter{section}
\stepcounter{section}
{\newpage
\clearpage
\samepage \begin{figure}% latex2html id marker 30569
[t]
\caption{Example of Logical Blocks, Conditional Newlines, and Sections}
\label{PRETTY-PRINT-SECTIONS-FIGURE}
\begin{lisp}
~~~~~~~~~~~~~~~~~<-1---<--<--2---3->--4-->-> \\ [4pt]
~~~~~~~~~~~~~~~~~000000000000000000000000000 \\ 
~~~~~~~~~~~~~~~~~11~111111111111111111111111 \\ 
~~~~~~~~~~~~~~~~~~~~~~~~~~~22~222            \\ 
~~~~~~~~~~~~~~~~~~~~~~~~~~~~~~333~3333       \\ 
~~~~~~~~~~~~~~~~~~~~~~~~44444444444444~44444
\end{lisp}
\end{figure}
}

\stepcounter{section}
\stepcounter{section}
\stepcounter{section}
\stepcounter{chapter}
\stepcounter{section}
\stepcounter{subsection}
\stepcounter{subsection}
{\newpage
\clearpage
\samepage ${\it S}_ 1$
}

{\newpage
\clearpage
\samepage ${\it S}_ 2$
}

{\newpage
\clearpage
\samepage ${\it S}_ 1\neq {\it S}_ 2$
}

{\newpage
\clearpage
\samepage ${\it C}_{1}$
}

{\newpage
\clearpage
\samepage ${\it C}_{2}$
}

{\newpage
\clearpage
\samepage ${\it C}_{\hbox{\scriptsize\it n}}$
}

{\newpage
\clearpage
\samepage ${\it C}_{2},\ldots,C_{\hbox{\scriptsize\it n}-1}$
}

{\newpage
\clearpage
\samepage ${\it C}_{\hbox{\scriptsize\it i}+1}$
}

{\newpage
\clearpage
\samepage ${\it C}_{\hbox{\scriptsize\it i}}$
}

{\newpage
\clearpage
\samepage ${\it C}_{1} \neq
C_{2}$
}

\stepcounter{subsubsection}
\stepcounter{subsubsection}
\stepcounter{subsubsection}
\stepcounter{subsubsection}
\stepcounter{subsection}
\stepcounter{subsubsection}
\stepcounter{subsubsection}
{\newpage
\clearpage
\samepage ${\it T}_ 1$
}

{\newpage
\clearpage
\samepage ${\it T}_ {\hbox{\scriptsize\it n}}$
}

{\newpage
\clearpage
\samepage ${\it T}_ 1, \ldots, T_ {\hbox{\scriptsize\it n}}$
}

\stepcounter{subsubsection}
\stepcounter{subsubsection}
\stepcounter{subsection}
{\newpage
\clearpage
\samepage \begin{table}% latex2html id marker 31776
[t]
\caption{Class Precedence Lists for Predefined Types}
\label{CLOS-PRECEDENCE-TABLE}
\begin{flushleft}
\cf
\begin{tabular}{@{}ll@{}}
{\rm Predefined Common Lisp Type}&{\rm Class Precedence List for Corresponding Class} \\ 
\hlinesp
array&(array t)\\ 
bit-vector&(bit-vector vector array sequence t)\\ 
character&(character t)\\ 
complex&(complex number t)\\ 
cons&(cons list sequence t)\\ 
float&(float number t)\\ 
function {\rm *}&(function t) \\ 
hash-table {\rm *}&(hash-table t) \\ 
integer&(integer rational number t)\\ 
list&(list sequence t)\\ 
null&(null symbol list sequence t)\\ 
number&(number t)\\ 
package {\rm *}&(package t) \\ 
pathname {\rm *}&(pathname t) \\ 
random-state {\rm *}&(random-state t) \\ 
ratio&(ratio rational number t)\\ 
rational&(rational number t)\\ 
readtable {\rm *}&(readtable t) \\ 
sequence&(sequence t)\\ 
stream {\rm *}&(stream t) \\ 
string&(string vector array sequence t)\\ 
symbol&(symbol t)\\ 
t&(t)\\ 
vector&(vector array sequence t)
\end{tabular}
\end{flushleft}
[An asterisk indicates a type added to this table as a consequence
of a portion of the CLOS specification that was conditional on X3J13 voting
to make that type disjoint from certain other built-in types
\issue{DATA-TYPES-HIERARCHY-UNDERSPECIFIED}.---GLS]
\end{table}
}

\stepcounter{subsection}
{\newpage
\clearpage
\samepage \begin{displaymath}{\it R}_ {\it C}=\{({\it C},{\it C}_ 1),({\it C}_ 1,{\it C}_ 2),\ldots,({\it C}_ {\hbox{\scriptsize\it n}-1},{\it C}_ {\hbox{\scriptsize\it n}})\}\end{displaymath}
}

{\newpage
\clearpage
\samepage ${\it C}_ 1,\ldots,{\it C}_ {\hbox{\scriptsize\it n}}$
}

{\newpage
\clearpage
\samepage ${\it S}_ {\it C}$
}

{\newpage
\clearpage
\samepage \begin{displaymath}{\it R}=\bigcup_{\textstyle {\it c}\in {{\it S}_ {\hbox{\scriptsize\it C}}}} {\it R}_ {\hbox{\scriptsize\it c}}\end{displaymath}
}

{\newpage
\clearpage
\samepage ${\it R}_ {\hbox{\scriptsize\it c}}$
}

{\newpage
\clearpage
\samepage ${\it c}\in {\it S}_ {\hbox{\scriptsize\it C}}$
}

{\newpage
\clearpage
\samepage ${\it S}_ {\hbox{\scriptsize\it C}}$
}

\stepcounter{subsubsection}
{\newpage
\clearpage
\samepage $({\it C},{\it D})$
}

{\newpage
\clearpage
\samepage ${\it D}\in {\it S}_ {\hbox{\scriptsize\it C}}$
}

{\newpage
\clearpage
\samepage ${\it C}_ 1,\ldots,C_ {\hbox{\scriptsize\it n}}$
}

{\newpage
\clearpage
\samepage $\{{\it N}_ 1,\ldots,{\it N}_ {\hbox{\scriptsize\it m}}\}$
}

{\newpage
\clearpage
\samepage $({\it C}_
1\ldots {\it C}_ {\hbox{\scriptsize\it n}})$
}

{\newpage
\clearpage
\samepage ${\it C}_
{\it n}$
}

{\newpage
\clearpage
\samepage ${\it N}_ {\hbox{\scriptsize\it i}}$
}

{\newpage
\clearpage
\samepage ${\it C}_ {\hbox{\scriptsize\it j}}$
}

{\newpage
\clearpage
\samepage ${\it T}_ 2$
}

\stepcounter{subsubsection}
{\newpage
\clearpage
\samepage ${\it S}=\{\cd{pie},\discretionary{}{}{}
\cd{apple},\discretionary{}{}{}
\cd{cinnamon},\discretionary{}{}{}
\cd{fruit},\discretionary{}{}{}
\cd{spice},\discretionary{}{}{}
\cd{food},\discretionary{}{}{}
\cd{standard{\copy\Qhyphbox}\discretionary{}{}{}object},\discretionary{}{}{}
\cd{t}\}$
}

{\newpage
\clearpage
\samepage ${\it R}=\{(\cd{pie},\discretionary{}{}{}
\cd{apple}),\discretionary{}{}{}
(\cd{apple},\discretionary{}{}{}
\cd{cinnamon}),\discretionary{}{}{}
(\cd{cinnamon},\discretionary{}{}{}
\cd{standard{\copy\Qhyphbox}\discretionary{}{}{}object}),\discretionary{}{}{}
(\cd{apple},\discretionary{}{}{}
\cd{fruit}),\discretionary{}{}{}
(\cd{fruit},\discretionary{}{}{}
\cd{standard{\copy\Qhyphbox}\discretionary{}{}{}object}),\discretionary{}{}{}
(\cd{cinnamon},\discretionary{}{}{}
\cd{spice}),\discretionary{}{}{}
(\cd{spice},\discretionary{}{}{}
\cd{standard{\copy\Qhyphbox}\discretionary{}{}{}object}),\discretionary{}{}{}
(\cd{fruit},\discretionary{}{}{}
\cd{food}),\discretionary{}{}{}
(\cd{food},\discretionary{}{}{}
\cd{standard{\copy\Qhyphbox}\discretionary{}{}{}object}),\discretionary{}{}{}
(\cd{spice},\discretionary{}{}{}
\cd{food}),\discretionary{}{}{}
(\cd{standard{\copy\Qhyphbox}\discretionary{}{}{}object},\discretionary{}{}{}
\cd{t})\}$
}

{\newpage
\clearpage
\samepage $(\cd{cinnamon},\discretionary{}{}{}
\cd{standard{\copy\Qhyphbox}\discretionary{}{}{}object})$
}

{\newpage
\clearpage
\samepage $(\cd{fruit},\discretionary{}{}{}
\cd{standard{\copy\Qhyphbox}\discretionary{}{}{}object})$
}

{\newpage
\clearpage
\samepage $(\cd{spice},\discretionary{}{}{}
\cd{standard{\copy\Qhyphbox}\discretionary{}{}{}object})$
}

{\newpage
\clearpage
\samepage ${\it S}=\{\cd{apple},\discretionary{}{}{}
\cd{cinnamon},\discretionary{}{}{}
\cd{fruit},\discretionary{}{}{}
\cd{spice},\discretionary{}{}{}
\cd{food},\discretionary{}{}{}
\cd{standard{\copy\Qhyphbox}\discretionary{}{}{}object},\discretionary{}{}{}
\cd{t}\}$
}

{\newpage
\clearpage
\samepage ${\it R}=\{
(\cd{apple},\discretionary{}{}{}
\cd{cinnamon}),\discretionary{}{}{}
(\cd{cinnamon},\discretionary{}{}{}
\cd{standard{\copy\Qhyphbox}\discretionary{}{}{}object}),\discretionary{}{}{}
(\cd{apple},\discretionary{}{}{}
\cd{fruit}),\discretionary{}{}{}
(\cd{fruit},\discretionary{}{}{}
\cd{standard{\copy\Qhyphbox}\discretionary{}{}{}object}),\discretionary{}{}{}
(\cd{cinnamon},\discretionary{}{}{}
\cd{spice}),\discretionary{}{}{}
(\cd{spice},\discretionary{}{}{}
\cd{standard{\copy\Qhyphbox}\discretionary{}{}{}object}),\discretionary{}{}{}
(\cd{fruit},\discretionary{}{}{}
\cd{food}),\discretionary{}{}{}
(\cd{food},\discretionary{}{}{}
\cd{standard{\copy\Qhyphbox}\discretionary{}{}{}object}),\discretionary{}{}{}
(\cd{spice},\discretionary{}{}{}
\cd{food}),\discretionary{}{}{}
(\cd{standard{\copy\Qhyphbox}\discretionary{}{}{}object},\discretionary{}{}{}
\cd{t})\}$
}

{\newpage
\clearpage
\samepage ${\it S}=\{\cd{cinnamon},\discretionary{}{}{}
\cd{fruit},\discretionary{}{}{}
\cd{spice},\discretionary{}{}{}
\cd{food},\discretionary{}{}{}
\cd{standard{\copy\Qhyphbox}\discretionary{}{}{}object},\discretionary{}{}{}
\cd{t}\}$
}

{\newpage
\clearpage
\samepage ${\it R}=\{(\cd{cinnamon},\discretionary{}{}{}
\cd{standard{\copy\Qhyphbox}\discretionary{}{}{}object}),\discretionary{}{}{}
(\cd{fruit},\discretionary{}{}{}
\cd{standard{\copy\Qhyphbox}\discretionary{}{}{}object}),\discretionary{}{}{}
(\cd{cinnamon},\discretionary{}{}{}
\cd{spice}),\discretionary{}{}{}
(\cd{spice},\discretionary{}{}{}
\cd{standard{\copy\Qhyphbox}\discretionary{}{}{}object}),\discretionary{}{}{}
(\cd{fruit},\discretionary{}{}{}
\cd{food}),\discretionary{}{}{}
(\cd{food},\discretionary{}{}{}
\cd{standard{\copy\Qhyphbox}\discretionary{}{}{}object}),\discretionary{}{}{}
(\cd{spice},\discretionary{}{}{}
\cd{food}),\discretionary{}{}{}
(\cd{standard{\copy\Qhyphbox}\discretionary{}{}{}object},\discretionary{}{}{}
\cd{t})\}$
}

{\newpage
\clearpage
\samepage ${\it S}=\{\cd{cinnamon},\discretionary{}{}{}
\cd{spice},\discretionary{}{}{}
\cd{food},\discretionary{}{}{}
\cd{standard{\copy\Qhyphbox}\discretionary{}{}{}object},\discretionary{}{}{}
\cd{t}\}$
}

{\newpage
\clearpage
\samepage ${\it R}=\{(\cd{cinnamon},\discretionary{}{}{}
\cd{standard{\copy\Qhyphbox}\discretionary{}{}{}object}),\discretionary{}{}{}
(\cd{cinnamon},\discretionary{}{}{}
\cd{spice}),\discretionary{}{}{}
(\cd{spice},\discretionary{}{}{}
\cd{standard{\copy\Qhyphbox}\discretionary{}{}{}object}),\discretionary{}{}{}
(\cd{food},\discretionary{}{}{}
\cd{standard{\copy\Qhyphbox}\discretionary{}{}{}object}),\discretionary{}{}{}
(\cd{spice},\discretionary{}{}{}
\cd{food}),\discretionary{}{}{}
(\cd{standard{\copy\Qhyphbox}\discretionary{}{}{}object},\discretionary{}{}{}
\cd{t})\}$
}

{\newpage
\clearpage
\samepage ${\it S}=\{\cd{spice},\discretionary{}{}{}
\cd{food},\discretionary{}{}{}
\cd{standard{\copy\Qhyphbox}\discretionary{}{}{}object},\discretionary{}{}{}
\cd{t}\}$
}

{\newpage
\clearpage
\samepage ${\it R}=\{(\cd{spice},\discretionary{}{}{}
\cd{standard{\copy\Qhyphbox}\discretionary{}{}{}object}),\discretionary{}{}{}
(\cd{food},\discretionary{}{}{}
\cd{standard{\copy\Qhyphbox}\discretionary{}{}{}object}),\discretionary{}{}{}
(\cd{spice},\discretionary{}{}{}
\cd{food}),\discretionary{}{}{}
(\cd{standard{\copy\Qhyphbox}\discretionary{}{}{}object},\discretionary{}{}{}
\cd{t})\}$
}

\stepcounter{subsection}
\stepcounter{subsubsection}
\stepcounter{subsubsection}
{\newpage
\clearpage
\samepage $\langle {\it A}_ 1, \ldots, {\it A}_ {\hbox{\scriptsize\it n}}\rangle$
}

{\newpage
\clearpage
\samepage $\langle {\it P}_ 1,
\ldots, {\it P}_ {\hbox{\scriptsize\it n}}\rangle$
}

{\newpage
\clearpage
\samepage ${\it A}_ {\hbox{\scriptsize\it i}}$
}

{\newpage
\clearpage
\samepage ${\it P}_ {\hbox{\scriptsize\it i}}$
}

{\newpage
\clearpage
\samepage $C={\it P}_ {\hbox{\scriptsize\it i}}$
}

\stepcounter{subsubsection}
{\newpage
\clearpage
\samepage ${\it P}_{1,1}\ldots P_{1,\hbox{\scriptsize\it n}}$
}

{\newpage
\clearpage
\samepage ${\it P}_{2,1}\ldots P_{2,\hbox{\scriptsize\it n}}$
}

{\newpage
\clearpage
\samepage ${\it P}_{1,\hbox{\scriptsize\it i}}$
}

{\newpage
\clearpage
\samepage ${\it P}_{2,\hbox{\scriptsize\it i}}$
}

{\newpage
\clearpage
\samepage ${\it P}_{1,\hbox{\scriptsize\it i}}=\hbox{{\tt(\cd{eql} $\hbox{{\it object}}_ 1$)}}$
}

{\newpage
\clearpage
\samepage ${\it P}_{2,\hbox{\scriptsize\it i}}=\hbox{{\tt(\cd{eql} $\hbox{{\it object}}_ 2$)}}$
}

{\newpage
\clearpage
\samepage $\hbox{{\it object}}_ 1$
}

{\newpage
\clearpage
\samepage $\hbox{{\it object}}_ 2$
}

\stepcounter{subsubsection}
\stepcounter{subsubsection}
\stepcounter{subsection}
\stepcounter{subsubsection}
\stepcounter{subsubsection}
\stepcounter{subsubsection}
\stepcounter{subsubsection}
{\newpage
\clearpage
\samepage ${\it M}_ 1,\ldots,{\it M}_ {\hbox{\scriptsize\it k}}$
}

{\newpage
\clearpage
\samepage $\langle {\it M}_ 1\;{\it a}_ 1\ldots {\it a}_ {\hbox{\scriptsize\it n}}\rangle$
}

{\newpage
\clearpage
\samepage $\langle {\it M}_ k\;{\it a}_ 1\ldots {\it a}_ {\hbox{\scriptsize\it n}}\rangle$
}

{\newpage
\clearpage
\samepage $\langle {\it M}_ {\hbox{\scriptsize\it i}} \;{\it a}_ 1\ldots {\it a}_
{\hbox{\scriptsize\it n}}\rangle$
}

{\newpage
\clearpage
\samepage ${\it M}_ i$
}

{\newpage
\clearpage
\samepage ${\it a}_ 1\ldots {\it a}_ {\hbox{\scriptsize\it n}}$
}

{\newpage
\clearpage
\samepage $\langle {\it M}_{\hbox{\scriptsize\it i}} \ {\it a}_ 1\ldots {\it a}_ {\hbox{\scriptsize\it n}}\rangle$
}

{\newpage
\clearpage
\samepage $\langle {\it M}_ {\hbox{\scriptsize\it j}} \ {\it a}_ 1\ldots {\it a}_ {\hbox{\scriptsize\it n}}\rangle$
}

\stepcounter{subsection}
\stepcounter{subsubsection}
\stepcounter{subsubsection}
\stepcounter{subsubsection}
\stepcounter{subsection}
\stepcounter{subsubsection}
\stepcounter{subsubsection}
\stepcounter{subsubsection}
\stepcounter{subsubsection}
{\newpage
\clearpage
\samepage \begin{tabular*}{\textwidth}{@{}l@{\extracolsep{\fill}}ll@{}}
&{\rm Defaulted Initialization}&{\rm Contents} \\ 
{\rm Form}&{\rm Argument List}&{\rm of Slot} \\ 
\hlinesp
\cd{(make-instance 'r)}&\cd{(a 1 b 2)}&\cd{1}\\ 
\cd{(make-instance 'r 'a 3)}&\cd{(a 3 b 2)}&\cd{3}\\ 
\cd{(make-instance 'r 'b 4)}&\cd{(b 4 a 1)}&\cd{4}\\ 
\cd{(make-instance 'r 'a 1 'a 2)}&\cd{(a 1 a 2 b 2)}&\cd{1} \\ 
\hline
\end{tabular*}
}

\stepcounter{subsubsection}
\stepcounter{subsubsection}
\stepcounter{subsubsection}
\stepcounter{subsection}
\stepcounter{subsubsection}
\stepcounter{subsubsection}
\stepcounter{subsubsection}
\stepcounter{subsubsection}
\stepcounter{subsection}
{\newpage
\clearpage
\samepage ${\it C}_ {\hbox{{\footnotesize\rm from}}}$
}

{\newpage
\clearpage
\samepage ${\it C}_ {\hbox{{\footnotesize\rm to}}}$
}

\stepcounter{subsubsection}
{\newpage
\clearpage
\samepage ${\it C}_
{\hbox{{\footnotesize\it to}}}$
}

{\newpage
\clearpage
\samepage ${\it C}_
{\hbox{{\footnotesize\it from}}}$
}

\stepcounter{subsubsection}
\stepcounter{subsubsection}
\stepcounter{subsection}
\stepcounter{subsubsection}
\stepcounter{section}
{\newpage
\clearpage
\samepage $\lbrack\!\lbrack\downarrow\!\hbox{{\it slot-option}}\,\rbrack\!\rbrack$
}

{\newpage
\clearpage
\samepage ${\it slot{\copy\hyphbox}entry}_ 1$
}

{\newpage
\clearpage
\samepage ${\it slot{\copy\hyphbox}entry}_ {\hbox{\scriptsize\it n}}$
}

{\newpage
\clearpage
\samepage ${\it declaration}_ 1$
}

{\newpage
\clearpage
\samepage ${\it declaration}_ {\hbox{\scriptsize\it m}}$
}

{\newpage
\clearpage
\samepage ${\it form}_ 1$
}

{\newpage
\clearpage
\samepage ${\it form}_ {\hbox{\scriptsize\it k}}$
}

{\newpage
\clearpage
\samepage ${\it variable{\copy\hyphbox}name}_ 1$
}

{\newpage
\clearpage
\samepage ${\it accessor{\copy\hyphbox}name}_ 1$
}

{\newpage
\clearpage
\samepage ${\it variable{\copy\hyphbox}name}_ {\hbox{\scriptsize\it n}}$
}

{\newpage
\clearpage
\samepage ${\it accessor{\copy\hyphbox}name}_ {\hbox{\scriptsize\it n}}$
}

{\newpage
\clearpage
\samepage ${\it Q}_ 1$
}

{\newpage
\clearpage
\samepage ${\it Q}_ {\hbox{\scriptsize\it n}}$
}

{\newpage
\clearpage
\samepage ${\hbox{{\it Q}}}_ {\hbox{\scriptsize\it j}}$
}

{\newpage
\clearpage
\samepage ${\it slot{\copy\hyphbox}entry}_ {\hbox{\scriptsize\it j}}$
}

{\newpage
\clearpage
\samepage ${\hbox{{\it slot-entry}}}_ {\hbox{\scriptsize\it j}}$
}

{\newpage
\clearpage
\samepage ${\it variable{\copy\hyphbox}name}_ {\hbox{\scriptsize\it j}}$
}

{\newpage
\clearpage
\samepage ${\it slot{\copy\hyphbox}name}_ {\hbox{\scriptsize\it j}}$
}

\stepcounter{chapter}
\stepcounter{section}
\stepcounter{section}
\stepcounter{section}
\stepcounter{subsection}
\stepcounter{subsection}
\stepcounter{subsection}
\stepcounter{subsection}
\stepcounter{subsection}
\stepcounter{subsection}
\stepcounter{subsection}
\stepcounter{subsection}
\stepcounter{subsection}
\stepcounter{subsection}
\stepcounter{subsection}
\stepcounter{subsection}
\stepcounter{subsection}
\stepcounter{subsection}
\stepcounter{subsection}
\stepcounter{subsection}
\stepcounter{subsection}
\stepcounter{subsection}
\stepcounter{section}
\stepcounter{subsection}
\stepcounter{subsection}
\stepcounter{subsection}
\stepcounter{subsection}
{\newpage
\clearpage
\samepage \({}_{\hbox{\scriptsize\it n}}\)
}

{\newpage
\clearpage
\samepage \({}_{\hbox{\scriptsize\it m}}\)
}

\stepcounter{subsection}
\stepcounter{subsection}
\stepcounter{subsection}
{\newpage
\clearpage
\samepage ${\it C}'$
}

\stepcounter{subsection}
\stepcounter{subsection}
\stepcounter{subsection}
\stepcounter{subsection}
\stepcounter{section}
{\newpage
\clearpage
\samepage \begin{table}% latex2html id marker 37644
[t]
\caption{Condition Type Hierarchy}
\label{CONDITION-HIERARCHY-TABLE}
\begin{lisp}
condition \\ 
~~~~simple-condition \\ 
~~~~serious-condition \\ 
~~~~~~~~error \\ 
~~~~~~~~~~~~simple-error \\ 
~~~~~~~~~~~~arithmetic-error \\ 
~~~~~~~~~~~~~~~~division-by-zero \\ 
~~~~~~~~~~~~~~~~floating-point-overflow \\ 
~~~~~~~~~~~~~~~~floating-point-underflow \\ 
~~~~~~~~~~~~~~~~... \\ 
~~~~~~~~~~~~cell-error \\ 
~~~~~~~~~~~~~~~~unbound-variable \\ 
~~~~~~~~~~~~~~~~undefined-function \\ 
~~~~~~~~~~~~~~~~... \\ 
~~~~~~~~~~~~control-error \\ 
~~~~~~~~~~~~file-error \\ 
~~~~~~~~~~~~package-error \\ 
~~~~~~~~~~~~program-error \\ 
~~~~~~~~~~~~stream-error \\ 
~~~~~~~~~~~~~~~~end-of-file \\ 
~~~~~~~~~~~~~~~~... \\ 
~~~~~~~~~~~~type-error \\ 
~~~~~~~~~~~~~~~~simple-type-error \\ 
~~~~~~~~~~~~~~~~... \\ 
~~~~~~~~~~~~... \\ 
~~~~~~~~storage-condition \\ 
~~~~~~~~... \\ 
~~~~warning \\ 
~~~~~~~~simple-warning \\ 
~~~~~~~~... \\ 
~~~~...
\end{lisp}
\vfill
\end{table}
}

\appendix
\stepcounter{chapter}
\stepcounter{section}
\stepcounter{section}
{\newpage
\clearpage
\samepage \({}_{\hbox{\scriptsize\it j}}\)
}

\stepcounter{subsection}
{\newpage
\clearpage
\samepage ${\it t}_1$
}

{\newpage
\clearpage
\samepage ${\it t}x_m$
}

{\newpage
\clearpage
\samepage ${\it t}_{\hbox{\scriptsize\it i}}$
}

{\newpage
\clearpage
\samepage ${\it t}_1, \ldots\,, {\it t}_{\hbox{\scriptsize\it m}}$
}

{\newpage
\clearpage
\samepage ${\it t}_{\hbox{\scriptsize\it m}}$
}

{\newpage
\clearpage
\samepage \({}_{0}\)
}

{\newpage
\clearpage
\samepage \({}_{(\hbox{\scriptsize\it j}-1)}\)
}

\stepcounter{subsection}
{\newpage
\clearpage
\samepage ${\it s}_{\hbox{\scriptsize\it i}}$
}

{\newpage
\clearpage
\samepage ${\it s}_1$
}

{\newpage
\clearpage
\samepage ${\it s}_{\hbox{\scriptsize\it n}}$
}

\stepcounter{subsection}
{\newpage
\clearpage
\samepage ${\it k}'$
}

{\newpage
\clearpage
\samepage \({\it k}'\)
}

{\newpage
\clearpage
\samepage ${it k}''$
}

{\newpage
\clearpage
\samepage \({}_{\hbox{\scriptsize\it k}''}\)
}

{\newpage
\clearpage
\samepage \({\it k}''\)
}

\stepcounter{subsection}
{\newpage
\clearpage
\samepage $\lfloor1+({\it l}-{\it m})/{\it n}\rfloor$
}

{\newpage
\clearpage
\samepage $({\it i}*{\it n}+{\it k})$
}

\stepcounter{subsection}
\stepcounter{subsection}
\stepcounter{section}
\stepcounter{subsection}
\stepcounter{subsection}
{\newpage
\clearpage
\samepage \begin{figure}% latex2html id marker 39202
[t]
\caption{A Constraint Cycle in a Series Expression}\label{SERIES-F1-FIGURE}
\vskip 5pc
\PostScriptFile{series-plot.ps}\relax
\hbox{\relax

\vbox to 0pt{\vskip -3.5pc\vskip 3pt\hbox to 0pt{\hskip 2.5pc\hskip -3pt\vbox to 0pt{\vss
\hbox to 0pt{\hss \tt scan\hss}\vss}\hss}\vss}\relax
\vbox to 0pt{\vskip -2.5pc\vskip 3pt\hbox to 0pt{\hskip 8.5pc\hskip -3pt\vbox to 0pt{\vss
\hbox to 0pt{\hss \tt sum\hss}\vss}\hss}\vss}\relax
\vbox to 0pt{\vskip -2.5pc\vskip 3pt\hbox to 0pt{\hskip 14pc\hskip -3pt\vbox to 0pt{\vss
\hbox to 0pt{\hss \tt series\hss}\vss}\hss}\vss}\relax
\vbox to 0pt{\vskip -3.5pc\vskip 3pt\hbox to 0pt{\hskip 19.5pc\hskip -3pt\vbox to 0pt{\vss
\hbox to 0pt{\hss \tt \#M/\hss}\vss}\hss}\vss}\relax
\vbox to 0pt{\vskip -3.5pc\vskip 3pt\hbox to 0pt{\hskip 25pc\hskip -3pt\vbox to 0pt{\vss
\hbox to 0pt{\hss \tt max\hss}\vss}\hss}\vss}}
\end{figure}
}

\stepcounter{subsection}
\stepcounter{subsection}
\stepcounter{section}
\stepcounter{chapter}
\stepcounter{section}
\stepcounter{section}
\stepcounter{section}
{\newpage
\clearpage
\samepage $t_1$
}

{\newpage
\clearpage
\samepage $t_2$
}

\stepcounter{section}
\stepcounter{chapter}
{\newpage
\clearpage
\samepage $^3$
}

\stepcounter{chapter}

\end{document}
